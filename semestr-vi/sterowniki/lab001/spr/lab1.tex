\documentclass[wide,a4paper,titlepage,12pt] {article}
\usepackage{polski}
\usepackage[utf8]{inputenc}
\usepackage{listings}
\usepackage{placeins}


\title{Sterowniki mikroprocesorowe w aplikacjach sieciowych}
\author{Tymon Tobolski (181037)\\ Jacek Wieczorek (181043)}

% Title page layout (fold)
\makeatletter
\renewcommand{\maketitle}{
\begin{titlepage}
  \begin{center}
    \vspace*{3cm}
    \LARGE \@title \par
    \vspace{2cm}
    \textit{\small Autor:}\par
    \normalsize \@author\par \normalsize
    \vspace{3cm}
    \textit{\small Prowadzący:}\par
    Dr inż. Jerzy Greblicki \par
    \vspace{2cm}
    Wydział Elektroniki\\ III rok\\ Cz TP 12.15 - 15.00\par

  \end{center}
\end{titlepage}
}
\makeatother
  \lstset{
    language=c++,
    basicstyle=\ttfamily\scriptsize,
    numbers=left,
    numberstyle=\scriptsize,
    stepnumber=10,
    numbersep=9pt,
    showspaces=false,
    showstringspaces=false,
    showtabs=false,
    breaklines=true,
  }

\begin{document}
\maketitle
  \section{Cel laboratorium}
  \paragraph{}
  Celem laboratorium było zapoznanie się z podstawowymi operacjami wejścia/wyjścia na mikrokontrolerze Atmega 328p.

  \section{Kod źródłowy}
  \lstinputlisting{../lab001.c}

  \section{Wnioski}
  \paragraph{}
  Operacje wejścia i wyjścia polegające na zapalaniu diod lub obsłudze przycisku nie są skomplikowanym zadaniem.
  Ważnym czynnikiem jest mechanizm \textit{pull up}, który zapobiega przekłamaniom wskazania czy przycisk jest wciśnięty, czy nie. Mozna to zrobić na dwa sposoby : systemowo (jak pokazane zostało na listingu programu), lub elektronicznie za pomocą kondensatora.
\end{document}
