\documentclass[wide,a4paper,titlepage,12pt] {article}
\usepackage{polski}
\usepackage[utf8]{inputenc}
\usepackage{listings}
\usepackage{placeins}


\title{Sterowniki mikroprocesorowe w aplikacjach sieciowych}
\author{Tymon Tobolski (181037)\\ Jacek Wieczorek (181043)}

% Title page layout (fold)
\makeatletter
\renewcommand{\maketitle}{
\begin{titlepage}
  \begin{center}
    \vspace*{3cm}
    \LARGE \@title \par
    \vspace{2cm}
    \textit{\small Autor:}\par
    \normalsize \@author\par \normalsize
    \vspace{3cm}
    \textit{\small Prowadzący:}\par
    Dr inż. Jerzy Greblicki \par
    \vspace{2cm}
    Wydział Elektroniki\\ III rok\\ Cz TP 12.15 - 15.00\par

  \end{center}
\end{titlepage}
}
\makeatother
  \lstset{
    language=c++,
    basicstyle=\ttfamily\scriptsize,
    numbers=left,
    numberstyle=\scriptsize,
    stepnumber=10,
    numbersep=9pt,
    showspaces=false,
    showstringspaces=false,
    showtabs=false,
    breaklines=true,
  }

\begin{document}
\maketitle
  \section{Cel laboratorium}
  \paragraph{}
  Celem laboratorium było zapoznanie się z obsługą liczników oraz przerwań na mikrokontrolerze Atmega 328p.

  \section{Kod źródłowy}
  \lstinputlisting{../lab002.c}

  \section{Czyszczenie flagi TOV1}
  \paragraph{}
  Operacja czyszczenia flagi TOV1 powinna być wykonywana tak szybko jak to
  jest możliwe stąd też rozwiązanie polegające na wpisywanie logicznej jedynki
  zapewnia najkrótszy czas wykonania zadania.
  \paragraph{} % (fold)

  Zapis logicznej 1 na pozycji TOV1 wymaga tylko jednej operacji,
  ponadto wpisanie 0 do pozostałych bitów rejestru TIFR1 nie zmieni ich wartości.
  W przypadku zapisu na pozycji TOV1 wartości 0 wymaga użycia również operacji odczytu, zmiany i zapisu,
  co może skutkować zmienieniem innych wartości w rejestrze.

  \section{Wnioski}
  \paragraph{}
  Operacje na rejestrach liczników nie są skomplikowanym zadaniem. Można je wykonać na kilka sposobów,
  bez lub z wykorzystaniem mechanizmu przerwań. Dzięki wykorzystaniu mechanizmu obsługi przerwań
  jesteśmy w stanie wykonywać operacje na licznikach \"w tle\" kontrolera, bez potrzeby blokowania
  wykonania innych operacji.
\end{document}
