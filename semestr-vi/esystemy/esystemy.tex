\documentclass[wide,a4paper,titlepage,12pt] {article}
\usepackage{polski}
\usepackage[utf8]{inputenc}
\usepackage{listings}
\usepackage{placeins}


\title{Inżynieria e-systemów java}
\author{Tymon Tobolski (181037)\\ Jacek Wieczorek (181043) \\ Mateusz Lenik(181142)}

% Title page layout (fold)
\makeatletter
\renewcommand{\maketitle}{
\begin{titlepage}
  \begin{center}
    \vspace*{3cm}
    \LARGE \@title \par
    \vspace{2cm}
    \textit{\small Autor:}\par
    \normalsize \@author\par \normalsize
    \vspace{3cm}
    \textit{\small Prowadzący:}\par
    Dr inż. Katarzyna Nowak \par
    \vspace{2cm}
    Wydział Elektroniki\\ III rok\\ Śr 7.30 - 9.00\par

  \end{center}
\end{titlepage}
}
\makeatother
  \lstset{
    language=c++,
    basicstyle=\ttfamily\scriptsize,
    numbers=left,
    numberstyle=\scriptsize,
    stepnumber=10,
    numbersep=9pt,
    showspaces=false,
    showstringspaces=false,
    showtabs=false,
    breaklines=true,
  }

\begin{document}
\maketitle
  \section{Opis projektu}
  \paragraph{}
  Celem projektu jest stworzenie portalu, pozwalającego użytkownikowi na zakładanie i porwadzenie mikroblogów wraz zmożliwością follow'ania wpisów innych użytkowników. Aplikacja zostanie napisana w języku ruby z wykorzystaniem framework'a Ruby on Rails.
  \begin{itemize}
    \item Platforma : JVM
    \item Język impementacji : ruby, coffescript
    \item Framework : Ruby on Rails 
  \end{itemize}
  \section{Funkcjonalności systemu}
  \paragraph{}
  \begin{itemize}
   \item Obsługa użytkowników
   \begin{itemize}
    \item Tworzenie konta
    \item Edycja użytkownika
    \item System autentykacji
    \item Usuwanie użytkowników
    \item Follow user
   \end{itemize}
   \item Obsługa blogów
    \begin{itemize}
      \item Tworzenie bloga
      \item Zarządzanie blogiem
      \item Tablica z postami użytkowników "follow"
    \end{itemize}
   \item Obsługa postów
    \begin{itemize}
      \item Dodawanie posta
      \item Usuwanie posta
    \end{itemize}
  \item Dyskusje
    \begin{itemize}
      \item Tworzenie dyskusji
      \item Zarządzanie dyskusjami
      \item Dyskusje publiczne i prywatne
      \item Zapraszanie użytkowników
      \item Dodawanie postów w dyskusji
    \end{itemize}
  \item Dodatkowe funkcjonalności
    \begin{itemize}
      \item Powiązanie kont użytkowników z kontami facebook
      \item Pobieranie avatarów z gravatara lub facebook (zdjęcie profilowe) 
    \end{itemize}
  \end{itemize}
  \section{Podział zadań}
  \paragraph{}
  W pracy nad projektem zamierzamy wykorzystać metodologię SCRUM, która zakłada podział projektu na małe zadania, nie moduły. Z tego względu nie jesteśmy wstanie określić kto będzie się zajmował jaką częścią projektu. Zadania będą w miarę postępu prac przypisane członkom zespołu.
  \paragraph{}
  Oprócz wyżej wymienionych funkcjonalności w projekcie należy również zadbać o frontend i testy całego systemu. 


  
\end{document}