\documentclass[wide,a4paper,titlepage,12pt] {article}
\usepackage{polski}
\usepackage[utf8]{inputenc}
\usepackage{listings}
\usepackage{slashbox}
\usepackage[table]{xcolor}
\usepackage{graphicx,pdflscape}
\usepackage{placeins}

\title{Projektowanie efektywnych algorytmów}
\author{Tymon Tobolski (181037)\\ Jacek Wieczorek (181043)}

% Title page layout (fold)
\makeatletter
\renewcommand{\maketitle}{
\begin{titlepage}
  \begin{center}
    \vspace*{3cm}
    \LARGE \@title \par
    \vspace{2cm}
    \textit{\small Autor:}\par
    \normalsize \@author\par \normalsize
    \vspace{3cm}
    \textit{\small Prowadzący:}\par
   Prof. dr hab. inż Adam Janiak \par
    \vspace{2cm}
    Wydział Elektroniki\\ III rok\\ Cz TN 13.15 - 15.00\par
    \vspace{4cm}
    \small \@date
  \end{center}
\end{titlepage}
}
\makeatother

\begin{document}
\maketitle

\section{Cel projektu}
\paragraph{}
  Celem projektu jest zaimplementowanie i przetestowanie metaheurystycznego algorytmu genetycznego dla problemu szeregowania zadań na jednym procesorze przy kryterium minimalizacji ważonej sumy opóźnień zadań.

\section{Opis problemu}
{\bf Jednoprocesorowy problem szeregowania zadań przy kryterium
minimalizacji ważonej sumy opóźnień zadań.}

\paragraph{}
Danych jest $n$ zadań (o numerach od 1 do $n$), które mają być wykonane bez przerwań przez pojedynczy procesor, mogący wykonywać co najwyżej jedno zadanie jednocześnie.
Każde zadanie j jest dostępne do wykonania w chwili zero, do wykonania wymaga $p_{j} > 0$ jednostek czasu oraz ma określoną wagę (priorytet) $w_{j} > 0$ i oczekiwany termin zakończenia
wykonywania $d_{j} > 0$. Zadanie $j$ jest spóźnione, jeżeli zakończy się wykonywać po swoim terminie $d_{j}$, a miarą tego opóźnienia jest wielkość $T_{j} = max(0, C_{j} - d_{j} )$, gdzie $C_{j}$ jest terminem zakończenia
wykonywania zadania $j$. Problem polega na znalezieniu takiej kolejności wykonywania zadań (permutacji) aby zminimalizować kryterium $TWT = \Sigma_{j=1}^{n} w_{j} T_{j}$.

\section{Opis algorytmu}
\paragraph{}


\newpage
\paragraph{}
Przebieg algorytmu :
\lstset{ %
    language=java,                % choose the language of the code
    basicstyle=\scriptsize,       % the size of the fonts that are used for the code
    numbers=left,                   % where to put the line-numbers
    numberstyle=\scriptsize,      % the size of the fonts that are used for the line-numbers
    stepnumber=10,                   % the step between two line-numbers. If it's 1 each line
                                    % will be numbered
    numbersep=9pt,                  % how far the line-numbers are from the code
    % backgroundcolor=\color{white},  % choose the background color. You must add \usepackage{color}
    showspaces=false,               % show spaces adding particular underscores
    showstringspaces=false,         % underline spaces within strings
    showtabs=false,                 % show tabs within strings adding particular underscores
    % frame=single,                 % adds a frame around the code
    % tabsize=2,                  % sets default tabsize to 2 spaces
    % captionpos=b,                   % sets the caption-position to bottom
    breaklines=true,                % sets automatic line breaking
    % breakatwhitespace=false,        % sets if automatic breaks should only happen at whitespace
    % title=\lstname,                 % show the filename of files included with \lstinputlisting;
                                    % also try caption instead of title
    % escapeinside={\%*}{*)},         % if you want to add a comment within your code
    % morekeywords={*,...}            % if you want to add more keywords to the set
    }
    \lstinputlisting{code/pseudokod.txt}
    \paragraph{}
    gdzie :
    \begin{itemize}
        \item F - funkcja kosztu/celu
        \item M - prawdopodobieństwo mutacji
        
    \end{itemize}
\newpage
\section{Implementacja}
\paragraph{}
Jezykiem implementacji algorytmu jest $Scala$ w wersji $2.9.1$ działająca na $JVM$.
\paragraph{}
\lstset{ %
    language=java,                % choose the language of the code
    basicstyle=\scriptsize,       % the size of the fonts that are used for the code
    numbers=left,                   % where to put the line-numbers
    numberstyle=\scriptsize,      % the size of the fonts that are used for the line-numbers
    stepnumber=10,                   % the step between two line-numbers. If it's 1 each line
                                    % will be numbered
    numbersep=9pt,                  % how far the line-numbers are from the code
    % backgroundcolor=\color{white},  % choose the background color. You must add \usepackage{color}
    showspaces=false,               % show spaces adding particular underscores
    showstringspaces=false,         % underline spaces within strings
    showtabs=false,                 % show tabs within strings adding particular underscores
    % frame=single,                 % adds a frame around the code
    % tabsize=2,                  % sets default tabsize to 2 spaces
    % captionpos=b,                   % sets the caption-position to bottom
    breaklines=true,                % sets automatic line breaking
    % breakatwhitespace=false,        % sets if automatic breaks should only happen at whitespace
    % title=\lstname,                 % show the filename of files included with \lstinputlisting;
                                    % also try caption instead of title
    % escapeinside={\%*}{*)},         % if you want to add a comment within your code
    % morekeywords={*,...}            % if you want to add more keywords to the set
    }
    \lstinputlisting{code/gen.scala}
\newpage
\section{Testy}
\paragraph{}
Test algorytmu tabu search przeprowadzony został dla trzech zestawów testów o różnej ilośći zadań, każdy składający się ze 125 instancji.

\paragraph{}
Jako wyniki testów przedstawiamy średni czas liczenia wszystkich instancji dla danego rozmiaru problemu - $\bar{t}$, a także średni błąd wzgledny  rozwiązań dla każdej instancji - $\bar{x}$. Według wzoru : \\
\begin{equation}
    \bar{t} = \frac{\Sigma_{j=1}^{m}\frac{\Sigma_{i=1}^{z}t_{i}}{z}}{m}
\end{equation}
\begin{equation}
    \bar{x} = \frac{\Sigma_{j=1}^{m}\frac{\Sigma_{i=1}^{z}x_{i}}{z}}{m}
\end{equation}
gdzie : \\
\begin{itemize}
  \item $z$ - ilość rozwiązań w instancji
  \item $m$ - ilość instancji danego problemu
\end{itemize}
\paragraph{}


\newpage
\section{Wnioski}
\paragraph{}
TODO
\end{document}



