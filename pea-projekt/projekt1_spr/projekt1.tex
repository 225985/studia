\documentclass[wide,a4paper,titlepage,12pt] {article}
\usepackage{polski}
\usepackage[utf8]{inputenc}
\usepackage{listings}
\usepackage{slashbox}
\usepackage[table]{xcolor}
\usepackage{graphicx,pdflscape}
\usepackage{placeins}

\title{Projektowanie efektywnych algorytmów}
\author{Tymon Tobolski (181037)\\ Jacek Wieczorek (181043)}

% Title page layout (fold)
\makeatletter
\renewcommand{\maketitle}{
\begin{titlepage}
  \begin{center}
    \vspace*{3cm}
    \LARGE \@title \par
    \vspace{2cm}
    \textit{\small Autor:}\par
    \normalsize \@author\par \normalsize
    \vspace{3cm}
    \textit{\small Prowadzący:}\par
   Prof. dr hab. inż Adam Janiak \par
    \vspace{2cm}
    Wydział Elektroniki\\ III rok\\ Cz TN 13.15 - 15.00\par
    \vspace{4cm}
    \small \@date
  \end{center}
\end{titlepage}
}
\makeatother

\begin{document}
\maketitle
  \section{Cel projektu}
\paragraph{}
Celem projketu jest zaimplementowanie i przetestowanie metaheurystycznego algorytmu symulowanego wyżarzania dla porblemu szeregowania zadań na jednym procesorze przy kryterium minimalizacji wazonej sumy opóźnień zadań.
  \section{Opis problemu}
{\bf Jednoprocesorowy problem szeregowania zadań przy kryterium
minimalizacji ważonej sumy opóźnień zadań.}
\paragraph{}
Danych jest $n$ zadań (o numerach od 1 do $n$), które mają być wykonane bez przerwań przez pojedynczy procesor, mogący wykonywać co najwyżej jedno zadanie jednocześnie.
Każde zadanie j jest dostępne do wykonania w chwili zero, do wykonania wymaga $p_{j} > 0$ jednostek czasu oraz ma określoną wagę (priorytet) $w_{j} > 0$ i oczekiwany termin zakończenia
wykonywania $d_{j} > 0$. Zadanie $j$ jest spóźnione, jeżeli zakończy się wykonywać po swoim terminie $d_{j}$, a miarą tego opóźnienia jest wielkość $T_{j} = max(0, C_{j} - d_{j} )$, gdzie $C_{j}$ jest terminem zakończenia
wykonywania zadania $j$. Problem polega na znalezieniu takiej kolejności wykonywania zadań (permutacji) aby zminimalizować kryterium $TWT = \Sigma_{j=1}^{n} w_{j} T_{j}$.
   \section{Implementacja}

\end{document}



