\documentclass[wide,a4paper,titlepage,12pt] {article}
\usepackage{polski}
\usepackage[utf8]{inputenc}
\usepackage{listings}
\usepackage{slashbox}
\usepackage[table]{xcolor}
\usepackage{graphicx,pdflscape}
\usepackage{placeins}


\title{Urządzenia peryferyjne}
\author{Tymon Tobolski (181037)\\ Jacek Wieczorek (181043)}

% Title page layout (fold)
\makeatletter
\renewcommand{\maketitle}{
\begin{titlepage}
  \begin{center}
    \vspace*{3cm}
    \LARGE \@title \par
    \vspace{2cm}
    \textit{\small Autor:}\par
    \normalsize \@author\par \normalsize
    \vspace{3cm}
    \textit{\small Prowadzący:}\par
    Dr inż. Jacek Mazurkiewicz \par
    \vspace{2cm}
    Wydział Elektroniki\\ III rok\\ Pn 8.15 - 11.00\par
    \vspace{4cm}
    \small \@date
  \end{center}
\end{titlepage}
}
\makeatother



\begin{document}
\maketitle

\section{Cel laboratorium}
\paragraph{}
Celem laboratorium było zapoznanie się z zasadą działania drukarek atramentowych, a także ich obsługi (zmiana czcionki, rozmiaru, koloru, stylu) za pomocą komend języka $PCL$.

\section{Nawiązanie połączenia z drukarką}
\paragraph{}
W celu połączenia się z drukarką atramentową za pomocą portu $LTP$ skorzystaliśmy z bilbioteki $fstream$ dostępnej w języku $cpp$. \\\\
Kod programu : 
\lstset{ %
    language=c++,                % choose the language of the code
    basicstyle=\scriptsize,       % the size of the fonts that are used for the code
    numbers=left,                   % where to put the line-numbers
    numberstyle=\scriptsize,      % the size of the fonts that are used for the line-numbers
    stepnumber=10,                   % the step between two line-numbers. If it's 1 each line 
                                    % will be numbered
    numbersep=9pt,                  % how far the line-numbers are from the code
    % backgroundcolor=\color{white},  % choose the background color. You must add \usepackage{color}
    showspaces=false,               % show spaces adding particular underscores
    showstringspaces=false,         % underline spaces within strings
    showtabs=false,                 % show tabs within strings adding particular underscores
    % frame=single,                 % adds a frame around the code
    % tabsize=2,                  % sets default tabsize to 2 spaces
    % captionpos=b,                   % sets the caption-position to bottom
    breaklines=true,                % sets automatic line breaking
    % breakatwhitespace=false,        % sets if automatic breaks should only happen at whitespace
    % title=\lstname,                 % show the filename of files included with \lstinputlisting;
                                    % also try caption instead of title
    % escapeinside={\%*}{*)},         % if you want to add a comment within your code
    % morekeywords={*,...}            % if you want to add more keywords to the set
    }
    \lstinputlisting{p1.cpp}

\section{Komendy języka $PCL$}
\paragraph{}
Język $PCL$ jest zbiorem komend escpowych opracowanym przez firmę \textit{Hewlett-Packard} jako protokół drukowania, który stał się pewnego rodzaju standardem.
\paragraph{}
Każda komenda języka $PCL$ zaczyna się od znaku $ESC$. Przed rozpoczęciem i po zakończeniu wysyłania komend formatujących tekst należy wysłać sekwencję resetującą drukarkę $ESC+E$
\subsection{Zmiana stylu czcionki}
\paragraph{}
Zmiana stylu czcionki ( kursywa, pogrubienie, podkreślenie, podwójne podkreślenie) odbywa się za pomocą następujących komend :
\begin{itemize}
	\item kursywa - $(s1S$ ,  wyłączenie - $(s0S$  
	\item pogrubienie - $(s3B$, wyłączenie - $(s0B$
	\item podkreślenie - $\&d0D$, wyłączenie - $\&d@$
	\item podwójne podkreślenie - $\&d2D$, wyłączenie - $\&d@$
\end{itemize} 

\subsection{Zmiana typu czcionki}
\paragraph{}
Język $PCL$ pozwala na wybór czcionek spośród zapisanych w pamięci drukarki, a także na wgrywanie i definiowanie własnych fontów. Pozwala również na tworzenie bardiej skomplikowanych kombinacji pozwalających na określeniu stylu czcionki, odstepów między znakami, rozmiarowi czcionki i jej typowie za pomocą jednej komendy np. $(s1p4101T$, która ustawia czcionkę \textit{GC Times New Roman} o rozmiarze \textit{12p} 

\subsection{Zmiana koloru czcionki}
\paragraph{}
By zmeinić kolor czcionki należy najpierw za pomocą komendy $*r3U$, ustawić wybór palety kolorów $RGB$. Następnie komendą $*vis$ wybieramy odpowiedni kolor z palety 8 podstawowych kolorów ($i$ - numer od 0 - 7).

\section{Program}
\paragraph{}
Program drukuje różne kombinacje komend escapowych pozwalających formatować tekst w dowolny sposób. Rezultat programu został zamieszczony w sprawozdaniu jako wynik laboratorium.
\lstset{ %
    language=c++,                % choose the language of the code
    basicstyle=\scriptsize,       % the size of the fonts that are used for the code
    numbers=left,                   % where to put the line-numbers
    numberstyle=\scriptsize,      % the size of the fonts that are used for the line-numbers
    stepnumber=10,                   % the step between two line-numbers. If it's 1 each line 
                                    % will be numbered
    numbersep=9pt,                  % how far the line-numbers are from the code
    % backgroundcolor=\color{white},  % choose the background color. You must add \usepackage{color}
    showspaces=false,               % show spaces adding particular underscores
    showstringspaces=false,         % underline spaces within strings
    showtabs=false,                 % show tabs within strings adding particular underscores
    % frame=single,                 % adds a frame around the code
    % tabsize=2,                  % sets default tabsize to 2 spaces
    % captionpos=b,                   % sets the caption-position to bottom
    breaklines=true,                % sets automatic line breaking
    % breakatwhitespace=false,        % sets if automatic breaks should only happen at whitespace
    % title=\lstname,                 % show the filename of files included with \lstinputlisting;
                                    % also try caption instead of title
    % escapeinside={\%*}{*)},         % if you want to add a comment within your code
    % morekeywords={*,...}            % if you want to add more keywords to the set
    }
    \lstinputlisting{p2.cpp}


\section{Wnioski}
\paragraph{}
Obsługa drukarki za pomocą komend języka $PCL$ nie jest rzeczą trudną i skomplikowaną. Wymaga od użytkownika znajomości podstawowych zasad wykonywania poleceń i umiejętności wyszukiwania potrzebnych komend w dokumentacji.




\end{document}