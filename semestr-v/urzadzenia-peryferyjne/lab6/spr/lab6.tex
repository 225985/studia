\documentclass[wide,a4paper,titlepage,12pt] {article}
\usepackage{polski}
\usepackage[utf8]{inputenc}
\usepackage{listings}
\usepackage{slashbox}
\usepackage[table]{xcolor}
\usepackage{graphicx,pdflscape}
\usepackage{placeins}
\usepackage{reportshelper}


\title{Urządzenia peryferyjne}
\author{Tymon Tobolski (181037)\\ Jacek Wieczorek (181043)}

% Title page layout (fold)
\makeatletter
\renewcommand{\maketitle}{
\begin{titlepage}
  \begin{center}
    \vspace*{3cm}
    \LARGE \@title \par
    \vspace{2cm}
    \textit{\small Autor:}\par
    \normalsize \@author\par \normalsize
    \vspace{3cm}
    \textit{\small Prowadzący:}\par
    Dr inż. Jacek Mazurkiewicz \par
    \vspace{2cm}
    Wydział Elektroniki\\ III rok\\ Pn 8.15 - 11.00\par
    \vspace{4cm}
    \small \@date
  \end{center}
\end{titlepage}
}
\makeatother



\begin{document}
\maketitle

\section{Cel laboratorium}
\paragraph{}
Celem laboratorium było zapoznanie się z zasadą pracy i obsługą skanera płaskiego. W tym celu wykorzystana została biblioteka WIA 1.0 i język programowania $C\#$.

\section{Parametry skanowania}
\paragraph{} % (fold)
Poza samym zeskanowaniem obrazu, udało nam sie również zmeiniać podstawowe parametry skanowania, takie jak:
\begin{itemize}
    \item czarnobiałe lub kolorowe
    \item rozdzielczość (DPI)
    \item kontrast
    \item jasność
\end{itemize}

\subsection{Rozdzielczość}
\paragraph{}
DPI (dots per inch) - liczba plamek przypadająca na cal, używane jako miara rozdzielczości drukarek, ploterów, skanerów itp. 
\paragraph{} % (fold)
\label{par:}
Najczęściej używa się rozdzielczości 72 dpi (skanowanie tekstu, rozdzielczość obrazków umieszczanych w Internecie) bądź 300 dpi (typowe skanowanie obrazków). Drukarka atramentowa ma od 300 do 600 dpi, a laserowa 600 - 1800. 
% paragraph  (end)

\paragraph{} % (fold)
\label{par:}

% paragraph  (end)
\subsection{Kontrast}
\paragraph{} % (fold)
\label{par:}
Kontrast jest róznicą w parametrach wizualnych obiektu, która pozwala rozróżnić jeden obiekt od drugiego. W postrzeganiu świata realnego, kontrast jest różnicą w kolorach i jasności obiektów w tym samym polu widzenia.  

\paragraph{} % (fold)
\label{par:}
W wielu sytuacjach używa się różnych formalnych definicji kontrastu. Jedną z najpopularniejszych i najczęściej spotykanych jest : 
\begin{equation}
    \frac{diff(luminance)}{avg(luminance)}
\end{equation}

\paragraph{}
Inne definicje : 
\begin{itemize}
    \item Kontrast Webera - $\frac{I-I_{b}}{I_{b}}$, gdzie $I$ i $I_{b}$ reprezentują luminancję obiektu i tła
    \item Kontrast Michelsona -  $\frac{I_{max}-I_{min}}{I_{max}+I_{min}}$, gdzie $I_{max}$ i $I_{min}$ reprezentują maksymalną i minimalną luminancję 
\end{itemize}
% paragraph  (end)

\subsection{Jasność}
\paragraph{} % (fold)
\label{par:}
Jasność jest atrybutem wizualnej percepcji, w której źródło wydaje się odbijać lub emitować światło.

\paragraph{} % (fold)
\label{par:}
W systemie kolorów RGB, jasność może być srednią sumy wartości poszczególnych kolorów :
\begin{equation}
    \mu = \frac{R+G+B}{3}
\end{equation}
\section{Implementacja}
\paragraph{} % (fold)
\label{par:}
Program do obsługi skanera płaskiego napisany został w języku $C\#$ z wykorzystaniem biblioteki WIA 1.0 .

\paragraph{} % (fold)
\label{}
W celu wyboru urządzenia skanującego skorzystaliśmy ze standardowego okna dialogowego dostepnego w bibliotece.

\putcode{code/p1.cs}{c++} 

\paragraph{}
Do inicializacji urządzenia i określenia parametrów skanowania : 
\putcode{code/p2.cs}{c++} 

\paragraph{} % (fold)
\label{par:}
Proces skanowania :
\putcode{code/p3.cs}{c++} 

\paragraph{} % (fold)
\label{par:}
Zapis obrazu do pliku :
\putcode{code/p4.cs}{c++}

\paragraph{} % (fold)
\label{par:}
Przykładowe okno programu :
\putpicture{img/program.PNG}{Przykładowe okno programu}{\textwidth}{}
% paragraph  (end)

% paragraph  (end)

\section{Wnioski}
Napisanie programu do obsługi skanera płaskiego z wykorzystaniem biblioteki WIA nie jest zadaniem trudnym. Ustawiając odpowiednie parametry skanowania (kontrast, jasność, dpi) jesteśmy wstanie otrzymać praktycznie dowolne obrazy, których jakość ograniczają tylko możliwości urządzenia.
\paragraph{}
\end{document}