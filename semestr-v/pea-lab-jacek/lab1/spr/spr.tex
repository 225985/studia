\documentclass[wide,a4paper,titlepage,12pt] {article}
\usepackage{polski}
\usepackage[UTF8]{inputenc}
\usepackage{listings}
\usepackage{slashbox}
\usepackage[table]{xcolor}
\usepackage{graphicx,pdflscape}
\usepackage{placeins}


\title{Projektowanie efektywnych algorytmów}
\author{Jacek Wieczorek (181043)}

% Title page layout (fold)
\makeatletter
\renewcommand{\maketitle}{
\begin{titlepage}
  \begin{center}
    \vspace*{3cm}
    \LARGE \@title \par
    \vspace{2cm}
    \textit{\small Autor:}\par
    \normalsize \@author\par \normalsize
    \vspace{3cm}
    \textit{\small Prowadzący:}\par
    mgr inż. Karolina Mokrzysz \par
    \vspace{2cm}
    Wydział Elektroniki\\ III rok\\ Pn TP 11.15 - 13.00\par
    \vspace{4cm}
    \small \@date
  \end{center}
\end{titlepage}
}
\makeatother
% Title page layout (end)



\begin{document}
\maketitle
  \section{Opis problemu}
{\bf Jednoprocesorowy problem szeregowania zadań przy kryterium
minimalizacji ważonej sumy opóźnień zadań.}
\paragraph{}
Danych jest $n$ zadań (o numerach od 1 do $n$), które mają być wykonane bez przerwań przez pojedynczy procesor, mogący wykonywać co najwyżej jedno zadanie jednocześnie.
Każde zadanie j jest dostępne do wykonania w chwili zero, do wykonania wymaga $p_{j} > 0$ jednostek czasu oraz ma określoną wagę (priorytet) $w_{j} > 0$ i oczekiwany termin zakończenia
wykonywania $d_{j} > 0$. Zadanie $j$ jest spóźnione, jeżeli zakończy się wykonywać po swoim terminie $d_{j}$, a miarą tego opóźnienia jest wielkość $T_{j} = max(0, C_{j} - d_{j} )$, gdzie $C_{j}$ jest terminem zakończenia
wykonywania zadania $j$. Problem polega na znalezieniu takiej kolejności wykonywania zadań (permutacji) aby zminimalizować kryterium $TWT = \Sigma_{j=1}^{n} w_{j} T_{j}$. \\\\
{\it *Opis zapożyczony z opracowania dr Krysiaka.}

\section{Opis algorytmu}
\subsection{Branch and Bound}
\paragraph{}
Branch and Bound jest algorytmem służącym do znajdowania optymalnych rozwiązań dla różnych problemów optymalizacyjnych, szczególnie dla optymalizacji dyskretnej i kombinatorycznej. Polega na systematycznej iteracji po wszystkich kandydujących solucjach porblemu, z których większość jest odrzucana po niespełnienu kryterium dolnego lub górnego ograniczenia.
\subsection{Użyte algorytmy}
\paragraph{}
W celu rozwiązania problemu szeregowania zadań przy kryterium minimalizacji ważonej sumy opóźnień zadań zaimplementowałem 3 metody eliminacji permutacji.
\subsubsection{Pierwsza metoda eliminacyjna}
\paragraph{}
Pierwsza metoda eliminacyjna polega na określeniu parametru $LB$ poprzez wyliczenie wartości początkowej kombinacji zadań. Gdy dla danej permutacji o $k$ elementach $k < n$, zachodzi warunek $LB >\Sigma_{j=1}^{k} w_{j} T_{j} $, wtedy algorytm przechodzi o poziom wyżej w drzewie permutacji, w przeciwnym przypadku rozwiązanie jest odrzucane.
\subsubsection{Druga metoda eliminacyjna}
\paragraph{}
Druga metoda eliminacyjna polega na zamianie $k$-tego zadnia w permutacji $k$-elementowej, $k < n$ na początku z zadaniem $k-1$, następnie $k-2$. Jeżeli w którymś z tych przypadków suma opóźnień bedzie mniejsza niż w przypadku pierwotnej permutacji, rozwiązanie to jest odrzucane.
\subsubsection{Trzecia metoda eliminacyjna}
\paragraph{}
Trzecia metoda eliminacyjna polega na zamianie $k$-tego zadnia w permutacji $k$-elementowej, z elementem $i$-tym, $i < k$, jeżeli $p_{i} > p_{k}$. Jeżeli suma opóźnień będzie mniejsza niż w pierwotnej permutacji, to odrzucamy to rozwiązanie.
\section{Implementacja i środowisko testowe}
\paragraph{}
Językiem implementacji algorytmów jest $C\#$, .NET Framework 4.0, środowisko programistyczne Microsoft Visual Studio 2010.  W celu maksymalnego przyśpieszenia wykonywanych algorytmów, porgram został gruntownie zbadany w $dotTrace'ie$ służacym do profilowania kodu. Wszystkie funkcje języka $C\#$ powodujące zbyt duży narzut czasowy, zostały usunięte.
\paragraph{}
Mimo, że porgram wyposażony został w interfejs graficzny, testy zostały przeprowadzone w aplikacji konsolowej (co przyśpiesza rozwiązywanie problemów). Testy dla przeglądu zupełnego i pierwszej metody eliminacyjnej zostały wykonane dla zestawów, zawierających do 11 zadań, natomiast druga i trecia metoda dla wszytskich zestawów. Dla każdego zestawu i każdej metody wykonane zostały po 3 testy, a prezentowany wynik jest ich średnią arytmetyczną. Jednostką czasu są $ms$. \newpage
\section{Analiza wyników}

\begin{center}
    \begin{tabular}{|c|c|c|c|c|}
    \hline
    lz & PZ & 1E & 2E & 3E \\ \hline
5 & 0 &0 &0 &0 \\ \hline
6 & 0 &0,66&0 &0\\ \hline
8 & 27,66 &58,66 &1 &1\\ \hline
9 & 263 &561,33 &5 &5\\ \hline
10 & 2958,66 &4880 &21 &16\\ \hline
11 & 34995,66 &40462,66 &63 &46\\ \hline
12 & $\infty$ &$\infty$ &230 &143\\ \hline
13 & $\infty$ &$\infty$ &592,66 &359,66\\ \hline
15 & $\infty$ &$\infty$ &4898,66 &2533,33\\ \hline
17 & $\infty$ &$\infty$ &61490,33&23592,66\\ \hline
18 & $\infty$ &$\infty$ &152486,66 &58963\\ \hline
   
    \end{tabular}

  \end{center}

\paragraph{}
Testy dla przegladu zupełnego i pierwszej eliminacji nie były przeprowadzone dla kompletów zawierających więcej niż 11 zadań z powodu zbyt długie czasu ich wykonywania. Na pierwszy rzut oka zaskoczeniem może wydawać się, iż pierwsza metoda eliminacyjna jest wolniejsza od przeglądu zupełnego. Dzieje się tak iż, liczony przez nas $LB$ nie jest optymalną wartością. Powoduje to wielekrotne używanie funkcji liczącej aktualny koszt, która z moich analiz w profilerze kodu okazałą się najbardziej czasochłonna (około 60-70\% czasu).  Poprawe działania algorytmu można by było osiagnąć poprawiając funkcję obliczającą kryterium $LB$. Problem znika, gdy spojrzymy na drugą część tabeli, gdzie prezentowane są wyniki dla dwóch następnych metod, dające bardzo dobre rezultaty w przyśpieszaniu rozwiązywania algorytmu. Funkcja korzystająca ze wsyztskich 3 metod eliminacyjnych jest zdecydowanie najszybsz i najbardziej efektywna.
\section{Wnioski}
\paragraph{}
Algorytm $B\&B$ daje w wiekszości przypadków wymierne korzyści w przyśpieszaniu rozwiązywania problemu $sNPh$, jednak wymaga on wielu zabiegów, metod i często skomplikowanych algorytmów w celu przeprowadzania procesów eliminacji. Nie mamy też gwarancji, że układ zadań, obliczone $LB$ czy $UP$, rzeczywiście przyśpieszą rozwiązanie problemu. W pesymistycznym przypadku, algorytm może dawać gorsze wyniki niż przegląd zupełny, a zakres jego używalności wynosi do około 30 zadań. 




	
\end{document}