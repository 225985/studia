\section{Analiza potrzeb użytkownika}
\paragraph{}
Przy projektowaniu sieci lokalnej dla tak duzej firmy informatycznej nalezy wziac pod uwagę bardzo
wiele czynników, ale przede wszystkim zapewnić ciagły dostęp do zasobów, a także jak największa predkość
łącza.
\subsection{Główne wymagania jakie stawiane są wobec tworzonej sieci}
\begin{enumerate}
	\item Możliwość przeprowadzania tele i wideokonferencji przy minimalizacji zakłóceń przy transmisji zadań
	\item Ciągła możliwość połączenia z serwerem
	\item Bez problemowy $dwonload$ i $upload$ kodu z serwera
	\item Przeglądanie witryn internetowych
	\item Współdzielenie plików miedzy komputerami, serwerami. Bez problemowa wymiana plików między stacjami używającymi systemów operacyjnych Linux i Mac OS, a stacjami używającymi Windows.
	\item Backup danych składowanych na serwerach
	\item Możliwość pracy zdalnej za pomocą Remote Desktop i ssh.
\end{enumerate}

\subsection{Bezpieczeństwo sieci}
\begin{enumerate}
	\item Konfiguracja Firewall
	\item Oprogramowanie antywirusowe
\end{enumerate}

\subsection{Tele i wideokonference}
\paragraph{}
Z racji świadczonych usług dla klientów międzynarodowych niezbędne jest zapewnienie odpowiedniej przepustowości sieci do prowadzenia tele oraz videokonferencji. Zalecana przez producenta oprogramowania (Skype) minimalna przepustowość łącza pozwalająca na prowadzenie telekonferencji wynosi 30/30 kb/s, jednak w przypadku większej ilości osób rozmawiających jednocześnie wymagane jest szybsze łącze, ok. 200/100 kb/s. Wideokonferencje wymagają znacznie szybszego połączenia. Minimalna prędkość podana przez producenta to 128/128 kb/s, jednak podobnie jak w przypadku telekonferencji większa ilość osób uczestniczących w wideokonferencji zwiększa wymagania łącza internetowego do ok 4/1 Mb/s.
\subsection{Sieć bezprzewodowa}
\paragraph{}
W każdej sali konferencyjnej znajduje się punkt dostępowy sieci bezprzewodowej  oferujący jedynie dostęp do Internetu i innych komputerów w obrębie tej sali. Ma to na celu zwiększenie bezpiczeństwa i zablokowanie dostępu do sieci wewnętrznej firmy osobom postronnym. Sieć bezprzewodowa wykoanna będzie w standardzie 802.11n, będącym całkowicie zgodnym z poprzednim standardem 802.11g. Uwierzytelnienie użytkowników podłączających się do sieci odbywać się będzie za pomocą szyfrowania $WPA2-PSK$.
\paragraph{}
Ze względu na charakter i wymagania pracy osób zajmujacych się produkcją oprogramowania dla urządzeń mobilnych, zachodzi potrzeba utworzenia bezpicznej sieci bezprzewodowej z dostępem do sieci wewnętrznej firmy. Sieć ta o ograniczonym zasięgu, dostępna będzie dla wybranych urządzeń o zautoryzowanych adresach $MAC$.

\subsection{Program antywirusowy}
\paragraph{}
W celu zabezpiecznenia stacji roboczych przed złośliwym oprogramowaniem, użyty zostanie program antywirusowy ESET Nod32. Jest to opragramowanie zapewniające duży poziom bezpieczeństwa, jednocześnie nie obciążając zbytnio systemu komputerowego. Kolejna zaletą jest możliwość instalacji go na systemach Linux.

\subsection{VLAN}
\paragraph{}
Biorąc pod uwagę specyfike działania firmy i dynamiczne przydzielanie zadań poszczególnym pracownikom, najlepszym rozwiązaniem będzie odseparowanie logicznej struktury sieci od struktury fizycznej za pomocą wirtualnych sieci LAN. Serwery i stacje robocze używane przez konkretną grupę korzystają z tej samej sieci VLAN. Pozwoli to na współpracę wielu osób w ramach jednej grupy niezależnie od ich położenia. Wirtualne sieci LAN znacznie ułatwiają przenoszenie stacji roboczych między podsieciami oraz dodawanie nowych stacji roboczych do instniejących już sieci. Usprawniają też nadzorowanie ruchu w sieci, a także poprawiają bezpieczeństwo.

\subsection{VPN}
\paragraph{}
Ze względu na możliwość pracy zdalnej, pracownicy muszą mieć dostęp do serwerów znajdujących się w siedzibie firmy. Mając na uwadze bezpieczeństwo danych sieć firmowa musi udostępniać usługę VPN. Daje to możliwość monitoringu i logowania dostępu do zasobów w bezpieczny sposób, niezależnie od fizycznej lokalizacji pracownika.

\subsection{Jakość usług sieciowych}
\paragraph{}
W celu zapewnienia jak najlepszej jakości usług sieciowych, odpowiednich przepustowości łącza, a także eliminacji przeciążenia infrastruktury sieciowej w firmie, zastosowane zostanie urządzenie służące do limitowania ruchu sieciowego (limiter). Pozwoli ono ustalić priorytety połączeń (tele i wideokonferencje - najwyższy, przeglądanie internetu najniższy), ustawić $QoS$ oraz pozwoli na filtrowanie ruchu sieciowego, blokowanie niebezpiecznych stron internetowych, czy ograniczy ściąganie nielegalnych plików.

\subsection{Minimalna wymagana przepustowość}
\paragraph{}
Szacując ruch sieciowy w firmie należy rozdzielić ruch wewnątrz sieci lokalnej oraz ruch do sieci zewnętrznej (Internet). W przypadku analizy wymaganej przepustowości na zewnątrz sieci trzeba uwzględnić wymagania, które stawia wykorzystywane oprogramowanie.

\paragraph{}
Szacowany dzienny przepływ danych w sieci wewnętrznej dla jednego pracownika wynosi ok. 200 Mb. 
Biorąc pod uwage fakt, iż serwerownia mieści się w budynku pierwszym, a w budynku drugim będzie pracować ok. 75 osób, można przyjąć założenie, że dzienny transfer pomiędzy budynkami wyniesie 15 Gb. 
Ruch sieciowy nie jest stały w ciągu dnia, ze względu na sytuacje losowe wymagające wysokiej przepustowości sieci (np. reinstalacja systemu, aktualizacja oprogramowania, tworzenie kopii zapasowych, pobieranie nowego oprogramowania). 
Z tego względu budynki powinny zostać połączone światłowodem.

\paragraph{}
Poniższa tabela przedstawia zalecane przez producenta oprogramowania parametry przepsutowości łącza dla pojedynczego użytkownika. 
W najgorszym hipotetycznym przypadku potrzebuje on przepustowości rzędu 11/7 Mb/s.
Takie zapotrzebowanie na łącze jest jednak bradzo mało prawdopodobne.
Mimo tego, należy wziąc pod uwagę możliwość prowadzenia kilku wideokonferencji w tym samym czasie bez znacznego ograniczania dostępu do Internetu reszcie pracowników.

\begin{center}
    \begin{tabular}{|c|c|c|}
    \hline
       & Download [Mb/s]                & Upload [Mb/s] \\ \hline
       Komunikator internetowy          & 0,1   & 0,1   \\ \hline
       Telekonferencje                  & 0,2   & 0,1   \\ \hline
       Wideokonferencje                 & 4     & 1     \\ \hline
       Program pocztowy                 & 1     & 0,5   \\ \hline
       Zdalny pulpit (TeamViewer, RD)   & 5     & 5     \\ \hline
       Przeglądanie internetu           & 1     & 0,5   \\ \hline
       \textbf{SUMA}					& \textbf{11,3}  & \textbf{7,2}	\\ \hline
   \end{tabular}
\end{center}

\paragraph{}
Podsumowując wymagania dotyczące przepsutowości sieci zalecane łącze internetowe powinno posiadać następujące parametry :
\begin{itemize}
	\item Download : 20 Mb/s
	\item Upload : 10 Mb/s
\end{itemize}

\paragraph{}
W celu zapewnienia ciągłości połączenia z siecią Internet zalecane jest wydzierżawienie łącza zapasowego o przepustowości 10/5 Mb/s.

\paragraph{}
W celu zapewnienia skalowalności sieci, w przypadku zwiekszenai zatrudnionej liczby pracowników, umowa powinna byc zawarta na czas nieokreślony. Daje to możliwość w każdej chwili zwiększenia przepustowości łącza do wymaganej, lub w przypadku redukcji kosztów na zminiejszenie.


\subsection{Okablowanie}
\paragraph{}


\begin{itemize}
  \item Zważając na fakt, iż dwie siedziby firmy znajdują się w pewnej odległości od siebie, a niezbędny jest stały i szybki dostep do serwerów znajdujących się w jednym z budynków połączenie między dwoma budnykami firmy będzię zrealizowane za pomocą światłowodu 10 Gb/s (przewidywany jest wzrost ruchu sieciowego związany z wytwarzaniem coraz bardziej skomplikowanego oprogramowania i zatrudnieniem większej liczby osób)

  \item Ze względu na fakt, iż główny ruch w sieci odbywa się między użytkownikiem, a serwerem, gdzie przechowywany jest kod i aplikacje testowe, połączenia pionowe powinny zapewniać większą przepustowość, niż połączenia  poziome. Ten typ połączeń wykonany zostanie za pomocą okablowania typu 1000Base-T Gigabit Ethernet, skrętka ekranowana kategori 6.

  \item Okablowanie poziomie zostanie zrealizowane w technologi 100Base-T Fast Ethernet, skrętka ekranowana UTP kategori 6. Decydujemy się na ten typ okablowania, ponieważ pojedynczy użytkownicy sieci, nie będą potrzebowali większej przepustowości niż oferowana przez ten typ połączenia

\end{itemize}