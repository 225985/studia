\documentclass[wide,a4paper,titlepage,12pt] {article}
\usepackage{polski}
\usepackage{float}
\usepackage[utf8]{inputenc}
\usepackage{listings}
\usepackage{slashbox}
\usepackage[table]{xcolor}
\usepackage{graphicx,pdflscape}
\usepackage{placeins}


\title{Technologie sieciowe 2}
\author{Tymon Tobolski (181037)\\ Jacek Wieczorek (181043)}

% Title page layout (fold)
\makeatletter
\renewcommand{\maketitle}{
\begin{titlepage}
  \begin{center}
    \vspace*{3cm}
    \LARGE \@title \par
    \vspace{2cm}
    \textit{\small Autor:}\par
    \normalsize \@author\par \normalsize
    \vspace{3cm}
    \textit{\small Prowadzący:}\par
    Dr inż. Arkadiusz Grzybowski\par
    \vspace{2cm}
    Wydział Elektroniki\\ III rok\\ Pn TN 11.15 - 13.00\par
    \vspace{4cm}
    \small \@date
  \end{center}
\end{titlepage}
}
\makeatother


\begin{document}
\maketitle
  \section{Cel laboratorium}
  \paragraph{}
  Celem ćwiczenia było opanowanie umijętności konfiguracji routingu statycznego oraz protokołu RIP w sieci LAN podzielonej na kilka domen rozgłoszeniowych.

  \section{Adresacja}

\begin{table}[h]
  \begin{center}
    \begin{tabular}{|c|c|c|c|}
      \hline
      Podsieć & Ilość hostów & Adres sieci & Maska \\
      \hline
      A & 254 & 10.0.193.0  & 255.255.255.0 \\
      B & 62  & 10.0.194.0  & 255.255.255.128 \\
      C & 254 & 10.0.192.0  & 255.255.255.0 \\
      D & 2   & 10.0.194.64 & 255.255.255.252 \\
      \hline
    \end{tabular}
  \end{center}
  \caption{Adresacja}
\end{table}

  \section{Podłączenie oraz konfiguracja routerów i stacji roboczych}
  \paragraph{}
  Routery R1 i R2 zostały podłączone kablami prostymi do przełączników. Stacje robocze połączone zostały z odpowiednimi przełącznikami również za pomoca kabli prostych. Połączenie między routerami zostało zestawione za pomocą kabla typu serial. Konfiguracja routerów odbyła się poprzez kabel konsolowy (rollover).

  \section{Konfiguracja}

  \subsection{Stacja robocza podsieci A}
\begin{table}[H]
  \begin{center}
    \begin{tabular}{|c|c|}
      \hline
      Adres IP & 10.0.193.2 \\
      \hline
      Maska podsieci & 255.255.255.0 \\
      \hline
      Brama domyślna & 10.0.193.1 \\
      \hline
    \end{tabular}
  \end{center}
  \caption{Konfiguracja stacji roboczej podsieci C}
\end{table}

  \subsection{Stacja robocza podsieci C}
\begin{table}[H]
  \begin{center}
    \begin{tabular}{|c|c|}
      \hline
      Adres IP & 10.0.192.2 \\
      \hline
      Maska podsieci & 255.255.255.0 \\
      \hline
      Brama domyślna & 10.0.192.1 \\
      \hline
    \end{tabular}
  \end{center}
  \caption{Konfiguracja stacji roboczej podsieci C}
\end{table}

  \subsection{Router R1}

  \begin{verbatim}
  # ustawienie nazwy routera
  hostname TymonTobolski

  # konfiguracja interfejsu FastEthernet0/0
  int f0/0
  ip address 10.0.193.1 255.255.255.0
  no shutdown

  # konfiguracja interfejsu Serial0/2/0
  int s0/2/0
  ip address 10.0.194.65 255.255.255.252
  no shutdown

  # konfiguracja routingu statycznego
  ip route 0.0.0.0 0.0.0.0 10.0.195.0
  ip route 10.0.192.0 255.255.255.0 serial0/2/0


  # sprawdzenie konfiguracji
  show ip int brief

  Interface       IP-Address  OK? Method  Status                  Protocol
  FastEthernet0/0 10.0.193.1  YES manual  up                      up
  FastEthernet0/1 unassigned  YES unset   administratively  down  down
  Serial0/2/0     10.0.194.65 YES manual  up                      up
  Serial0/2/1     unassigned  YES unset   administratively  down  down


  # tabele routingu
  show ip route

  Gateway of last resort is 10.0.195.0 to network 0.0.0.0

      10.0.0.0/8 is variably subnetted, 4 subnets, 2 masks
  C     10.0.194.64/30 is directly connected. Serial0/2/0
  C     10.0.195.0/24 is directly connected, Loopback0
  C     10.0.192.0/24 is directly connected, Serial0/2/0
  C     10.0.193.0/24 is directly connected, FastEthernet0/0
  S*  0.0.0.0/0 [1/0] via 10.0.195.0
 \end{verbatim}

  \subsection{Router R2}

  \begin{verbatim}
  # ustawienie nazwy routera
  hostname JacekWieczorek

  # konfiguracja interfejsu FastEthernet0/0
  int f0/0
  ip address 10.0.192.1 255.255.255.0
  no shutdown

  # konfiguracja interfejsu Serial0/2/0
  int s0/2/0
  ip address 10.0.194.66 255.255.255.252
  no shutdown

  # konfiguracja routingu statycznego
  ip route 0.0.0.0 0.0.0.0 10.0.195.0
  ip route 10.0.193.0 255.255.255.0 serial0/2/0


  # sprawdzenie konfiguracji
  show ip int brief

  Interface       IP-Address  OK? Method  Status                  Protocol
  FastEthernet0/0 10.0.192.1  YES manual  up                      up
  FastEthernet0/1 unassigned  YES unset   administratively  down  down
  Serial0/2/0     10.0.194.66 YES manual  up                      up
  Serial0/2/1     unassigned  YES unset   administratively  down  down


  # tabele routingu
  show ip route

  Gateway of last resort is 10.0.195.0 to network 0.0.0.0

      10.0.0.0/8 is variably subnetted, 4 subnets, 2 masks
  C     10.0.194.64/30 is directly connected. Serial0/2/0
  C     10.0.195.0/24 is directly connected, Loopback0
  C     10.0.193.0/24 is directly connected, Serial0/2/0
  C     10.0.192.0/24 is directly connected, FastEthernet0/0
  S*  0.0.0.0/0 [1/0] via 10.0.195.0
  \end{verbatim}



  \subsection{Weryfikacja łączności}
  \paragraph{}
  Poprawność podłącznia i konfiguracji urządzeń została przeprowadzona za pomocą polecenia \textbf{ping} między następującymi hostami. Poniższa tabela przedstawia wyniki weryfikacji.

\begin{table}[h]
  \begin{center}
    \begin{tabular}{|c|c||c|c||c|}
      \hline
      Host początkowy & Adres IP & Host docelowy & Adres IP & Wynik \\
      \hline
      Stacja robocza p. A       & 10.0.193.2    & Router R1             & 10.0.193.1  & Połączno \\
      Router R1                 & 10.0.193.1    & Stacja robocza p. A   & 10.0.193.2  & Połączno \\

      Stacja robocza p. A       & 10.0.193.2    & Router R2             & 10.0.192.1  & Połączno \\
      Router R2                 & 10.0.192.1    & Stacja robocza p. A   & 10.0.193.2  & Połączno \\

      Stacja robocza p. C       & 10.0.192.2    & Router R1             & 10.0.193.1  & Połączno \\
      Router R1                 & 10.0.193.1    & Stacja robocza p. C   & 10.0.192.2  & Połączno \\

      Stacja robocza p. C       & 10.0.192.2    & Router R2             & 10.0.192.1  & Połączno \\
      Router R2                 & 10.0.192.1    & Stacja robocza p. C   & 10.0.192.2  & Połączno \\

      Stacja robocza p. A       & 10.0.193.2    & Stacja robocza p. C   & 10.0.192.2  & Połączno \\
      Stacja robocza p. C       & 10.0.192.2    & Stacja robocza p. A   & 10.0.193.2  & Połączno \\

      Router R1                 & 10.0.194.65   & Router R2             & 10.0.195.66 & Połączno \\
      Router R2                 & 10.0.194.66   & Router R1             & 10.0.194.65 & Połączno \\
      \hline
    \end{tabular}
  \end{center}
  \caption{Wyniki weryfikacji połączenia między urządzeniami}
\end{table}

  \paragraph{}
  Routery komunikują się bez problemów zarówno ze sobą, jak i ze stacjami roboczymi w obu podsieciach. Również stacje robocze mają możliwość połącznie do siebie nawzajem.

  \section{Konfiguracja RIP}
  \paragraph{}
  Poniżej znajduje się konfiguracja protokołu RIP.

  \subsection{Router R1}
  \begin{verbatim}
    router rip            # uruchomienie procesu rip
    version 2
    network 10.0.193.0    # dodatnie sieci
    network 10.0.194.64   # dodatnie sieci
    netowrk 10.0.195.0    # dodatnie sieci
    no auto-summary
    redistribute static   # rozgłaszanie tras statycznych
  \end{verbatim}

  \subsection{Router R2}
  \begin{verbatim}
    router rip            # uruchomienie procesu rip
    version 2
    network 10.0.192.0    # dodatnie sieci
    network 10.0.194.64   # dodatnie sieci
    no auto-summary
    redistribute static   # rozgłaszanie tras statycznych
  \end{verbatim}


  \section{Wnioski}
  \paragraph{}
  Routery Cisco pozwalają zarówno na dynamiczną jak i statyczną konfigurację routingu. Konfiguracja statyczna zapewnia szybszy routing jednak wymaga większego nakładu pracy od administratora oraz jest podatna na awarie.

  \paragraph{}
  Po skonfigurowanie routerów jak i stacji roboczych wykonane zostało sprawdzenie połączeń między urządzeniami - wszystkie połączenia okazały się być poprawne.

  \paragraph{}
  Ze względu na brak czasu nie udało się przeprowadzić weryfikacji ustawień protokołu RIP.

\end{document}