\documentclass[wide,a4paper,titlepage,12pt] {article}
\usepackage{polski}
\usepackage{float}
\usepackage[utf8]{inputenc}
\usepackage{listings}
\usepackage{slashbox}
\usepackage[table]{xcolor}
\usepackage{graphicx,pdflscape}
\usepackage{placeins}
\usepackage[final]{pdfpages}
\usepackage{longtable}
\usepackage{placeins}
\usepackage{url}

\title{Bazy danych 2 - projekt}
\author{Tymon Tobolski (181037)\\ Jacek Wieczorek (181043)}

% Title page layout (fold)
\makeatletter
\renewcommand{\maketitle}{
\begin{titlepage}
  \begin{center}
    \vspace*{3cm}
    \LARGE \@title \par
    \vspace{2cm}
    \textit{\small Autor:}\par
    \normalsize \@author\par \normalsize
    \vspace{3cm}
    \textit{\small Prowadzący:}\par
    Mgr inż. Mariusz Słabicki \par
    \vspace{2cm}
    Wydział Elektroniki\\ III rok\\ Cz 09.15 - 11.00\par
    \vspace{4cm}
    \small \@date
  \end{center}
\end{titlepage}
}
\makeatother

\begin{document}
\maketitle
  
\section{Cel projektu}
\paragraph{}
Celem projektu jest zaprojektowanie i zimplementowanie systemu do zarządzania projektem opartego na obiektowych bazach danych.

\section{Założenia projektu}
\paragraph{}
System do zarządzania projketami IT będzie oferował następujące funkcjonalności : możliwość dodawania użytkowników wraz ze statusami, prawa poszczególnych użytkowników, harmonogramowanie zadań, listy TODO, punkty kontrolne/kamienie milowe. Projekt oparty jest na obiektowej bazie danych Neodatis \footnote{Obiektowa baza danych Neodatis - \url{http://neodatis.org}}, zaimplementowanej w języku Java, interfejs użytkownika w Ruby on Rails \footnote{Framework sieciowy Ruby on Rails - \url{http://rubyonrails.org/}}. 

\section{Opis funkcjonalności}
  \begin{itemize}
    \item Tworzenie projektów
      \begin{itemize}
        \item Definiowanie nazwy, terminów, osób odpowiedzialnych, kamieni milowych, budżetu
      \end{itemize}

    \item Dodawanie nowych zadań do wykonania
      \begin{itemize}
        \item Przypisanie do projekt/kaminienia milowego
        \item Zdefiniowanie osoby odpowiedzialnej za zadanie
        \item Estymacja i logowanie czasu i kosztu spędzonego nad zadaniem
      \end{itemize}

   \item Zarządzanie zadaniami do wykonania
     \begin{itemize}
       \item Edycja zadań
       \item Dodawanie komentarzy i plików
       \item Zmiana stanu 
     \end{itemize}
     
    \item Zarządzanie uprawnieniami użytkowników
      \begin{itemize}
        \item Dostęp do projektów
        \item Pozwolenie na edycje / tylko do wglądu (konto klienta)
        \item Komentowanie zadań
        \item Publikacja załączników
      \end{itemize}

    \item System powiadomień o terminach
      \begin{itemize}
        \item Powiadomienie o nadchodzących terminach zadań, kamieni milowych, całego projektu
      \end{itemize}

    \item Statystyka
      \begin{itemize}
        \item Przebieg czasowy zadań (czas rozpoczęcia, zakończenia)
        \item Opóźnienia względem terminów
        \item Bilans zysków i strat (budżet)
      \end{itemize}

  \end{itemize}


\end{document}