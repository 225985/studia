\section{Analiza potrzeb użytkownika}
\paragraph{}
Przy projektowaniu sieci lokalnej dla tak duzej firmy informatycznej nalezy wziac pod uwagę bardzo
wiele czynników, ale przede wszystkim zapewnić ciagły dostęp do zasobów, a także jak największa predkość
łącza.
\subsection{Główne wymagania jakie stawiane są wobec tworzonej sieci}
\begin{enumerate}
	\item Możliwość przeprowadzania tele i wideokonferencji przy minimalizacji zakłóceń przy transmisji zadań
	\item Ciągła możliwość połączenia z serwerem
	\item Bez problemowy $dwonload$ i $upload$ kodu z serwera
	\item Przeglądanie witryn internetowych
	\item Współdzielenie plików miedzy komputerami, serwerami. Bez problemowa wymiana plików między stacjami używającymi systemów operacyjnych Linux i Mac OS, a stacjami używającymi Windows.
	\item Backup danych składowanych na serwerach
	\item Możliwość pracy zdalnej za pomocą Remote Desktop i ssh.
\end{enumerate}

\subsection{Bezpieczeństwo sieci}
\begin{enumerate}
	\item Konfiguracja Firewall
	\item Oprogramowanie antywirusowe
	\item Urządzenie limitujące ruch sieciowy
\end{enumerate}

\subsection{Tele i wideokonference}
\paragraph{}
Z racji świadczonych usług dla klientów midzynarodowych niezbędne jest zapewnienie odpowiedniej przepustowości sieci do prowadzenia tele oraz videokonferencji.
\subsection{Sieć bezprzewodowa}
\paragraph{}
W każdej sali konferencyjnej znajduje się punkt dostępowy sieci bezprzewodowej  oferujący jedynie dostęp do Internetu i innych komputerów w obrębie tej sali. Ma to na celu zwiększenie bezpiczeństwa i zablokowanie dostępu do sieci wewnętrznej firmy osobom postronnym. Sieć bezprzewodowa wykoanna będzie w standardzie 802.11g, będącym całkowicie zgodnym z poprzednim standardem 802.11b. Uwierzytelnienie użytkowników podłączających się do sieci odbywać się będzie za pomocą szyfrowania $WPA-PSK$.
\paragraph{}
Ze względu na charakter i wymagania pracy osób zajmujacych się produkcją oprogramowania dla urządzeń mobilnych, zachodzi potrzeba utworzenia bezpicznej sieci bezprzewodowej z dostępem do sieci wewnętrznej firmy. Sieć ta o ograniczonym zasięgu, dostępna będzie dla wybranych urządzeń o zautoryzowanych adresach $MAC$.

\subsection{Program antywirusowy}
\paragraph{}
W celu zabezpiecznenia stacji roboczych przed złośliwym oprogramowaniem, użyty zostanie program antywirusowy ESET Nod32. Jest to opragramowanie zapewniające duży poziom bezpieczeństwa, jednocześnie nie obciążając zbytnio systemu komputerowego. Kolejna zaletą jest możliwość instalacji go na systemach Linux.

\subsection{VLAN}
\paragraph{}
Biorąc pod uwagę specyfike działania firmy i dynamiczne przydzielanie zadań poszczególnym pracownikom, najlepszym rozwiązaniem będzie odseparowanie logicznej struktury sieci od struktury fizycznej za pomocą wirtualnych sieci LAN. Serwery i stacje robocze używane przez konkretną grupę korzystają z tej samej sieci VLAN. Pozwoli to na współpracę wielu osób w ramach jednej grupy niezależnie od ich położenia. Wirtualne sieci LAN znacznie ułatwiają przenoszenie stacji roboczych między podsieciami oraz dodawanie nowych stacji roboczych do instniejących już sieci. Usprawniają też nadzorowanie ruchu w sieci, a także poprawiają bezpieczeństwo.

\subsection{VPN}
\paragraph{}
Ze względu na możliwość pracy zdalnej, pracownicy muszą mieć dostęp do serwerów znajdujących się w siedzibie firmy. Mając na uwadze bezpieczeństwo danych sieć firmowa musi udostępniać usługę VPN. Daje to możliwość monitoringu i logowania dostępu do zasobów w bezpieczny sposób, niezależnie od fizycznej lokalizacji pracownika.

\subsection{Jakość usług sieciowych}
\paragraph{}
W celu zapewnienia jak najlepszej jakości usług sieciowych, odpowiednich przepustowości łącza, a także eliminacji przeciążenia infrastruktury sieciowej w firmie, zastosowane zostanie urządzenie służące do limitowania ruchu sieciowego (limiter). Pozwoli ono ustalić priorytety połączeń (tele i wideokonferencje - najwyższy, przeglądanie internetu najniższy), ustawić $QoS$ oraz pozwoli na filtrowanie ruchu sieciowego, blokowanie niebezpiecznych stron internetowych, czy ograniczy ściąganie nielegalnych plików.

\subsection{Minimalna wymagana przepustowość}
\paragraph{}
Szacując ruch sieciowy w firmie należy rozdzielić ruch wewnątrz sieci lokalnej oraz ruch do sieci zewnętrznej (Internet). W przypadku analizy wymaganej przepustowości na zewnątrz sieci trzeba uwzględnić wymagania, które stawia wykorzystywane oprogramowanie.

Analizując profil oferowanych usług przez firmę, wykorzystywane oprogramowanie oraz specyfike branży

\begin{center}
    \begin{tabular}{|c|c|c|}
    \hline
       & Download [Mb/s]                & Upload [MB/s] \\ \hline
       Komunikator internetowy          & 0,1   & 0,1   \\ \hline
       Telekonferencje                  & 0,1   & 0,1   \\ \hline
       Wideokonferencje                 & 2     & 0,5   \\ \hline
       Program pocztowy                 & 1     & 0,5   \\ \hline
       Zdalny pulpit (TeamViewer, RD)   & 5     & 5     \\ \hline
       System kontroli wersji           & 1     & 0,5   \\ \hline
       Przeglądanie internetu           & 1     & 0,5   \\ \hline
   \end{tabular}
\end{center}

\subsection{Okablowanie}
\paragraph{}


\begin{itemize}
  \item Połączenie między dwoma budnykami firmy będzię zrealizowane za pomocą światłowodu 10 Gb/s

  \item Ze względu na fakt, iż główny ruch w sieci odbywa się między użytkownikiem, a serwerem, gdzie przechowywany jest kod i aplikacje testowe, połączenia pionowe powinny zapewniać większą przepustowość, niż połączenia  poziome. Ten typ połączeń wykonany zostanie za pomocą okablowania typu 1000Base-T Gigabit Ethernet, skrętka ekranowana kategori 6.

  \item Okablowanie poziomie zostanie zrealizowane w technologi 100Base-T Fast Ethernet, skrętka foliowana UTP kategori 6. Decydujemy się na ten typ okablowania, ponieważ pojedynczy użytkownicy sieci, nie będą potrzebowali większej przepustowości niż oferowana przez ten typ połączenia

\end{itemize}