\section{Inwentaryzacja sprzętu i infrastruktury dostępnej w przedsiębiorstwie}

Z racji świadczonych usług dla klientów midzynarodowych niezbędne jest zapewnienie odpowiedniej przepustowości sieci do prowadzenia tele oraz videokonferencji.


\subsection{Telekonferencje}


\subsection{Videokonferencje}


\subsection{Wyposażenie}
\paragraph{}
Wyposażeniem każdego pracownika jest stacjonarny zestaw komputerowy, w skład którego wchodzą: jednostka centralna, mysz, klawiatura, monitor, kamera internetowa, słuchawki z mikrofonem.
Na każdym piętrze znajduje się sieciowe urządzenie wielofunkcyjne, podłączone i skonfigurowane w sposób zapewniający dostęp wszystkim pracownikom z danego piętra.

\paragraph{}
Każda z sal konferencyjnych została wyposażona w rzutnik multimedialny, a także komputer stacjonarny umożliwiający prowadzenie tele i wideokonferencji.
Ponadto w każdej z sal umieszczony jest punkt dostępowy sieci bezprzewodowej.

\paragraph{}
Część parteru jednego z budynków została zaadaptowana jako serwerownia, w której umieszczono kilka serwerów. Serwery te pozwalają na przechowywanie repozytowiów kodu źródłowego, przprowadzanie testów oprogramowania, składownie i wymianę plików między pracownikami, kopie zapasowe danych, a także dostęp do baz danych wykorzystywanych do administracji oraz przy pracy nad projektami.

\paragraph{}
Istotne dla projektu sieci oprogramowanie wykorzystywane przez pracowników:
\begin{itemize}
  \item Komunikator internetowy (protokół XMPP)
  \item Program do tele i videokonferencji Skype
  \item Pogram pocztowy (dowolny)
  \item System kontroli wersji (svn, git)
  \item Oprogramowanie umożliwiające współdzielenie plików Samba
\end{itemize}

\paragraph{}
Systemy operacyjne dostępne dla pracowników:
\begin{itemize}
  \item Windows 7
  \item Ubuntu 11
\end{itemize}

\paragraph{}
Systemy operacyjne zainstalowane na serwerach:
\begin{itemize}
  \item Windows Server 2008 IIS 7.0
  \item Ubuntu Server 11
\end{itemize}




