\section{Inwentaryzacja sprzętu i infrastruktury dostępnej w przedsiębiorstwie}
\subsection{Budynki}
\paragraph{}
Firma ma swoją siedzibę w dwóch budynkach oddalonych od siebie o około 50m. Pierwsza z budowli składa się z czterech pieter, natomiast druga z trzech. Sześć kondygnacji jest zaadaptowanych jako pomieszczenia dla programistów. Ostatnia nieużywana kondygnacja mieścic bedzie w sobie serwerownie, pomieszczenia administracyjne, a także pomieszczenia członków zarządu. Na każdym piętrze zlokalizowana będzie sala konferencyjna, oraz kuchnia i pomieszczenia sanitarne.
\paragraph{}
\subsubsection {Budynke 1} 
TODO co sie w nim znajduje na kazdym pietrze \\

\subsubsection{Budynke 2} 
TODO co sie w nim znajduje na kazdym pietrze \\

\subsection{Wyposażenie}
\paragraph{}
Wyposażeniem każdego pracownika jest stacjonarny zestaw komputerowy, w skład którego wchodzą: jednostka centralna, mysz, klawiatura, monitor, kamera internetowa, słuchawki z mikrofonem.
Na każdym piętrze znajduje się sieciowe urządzenie wielofunkcyjne, podłączone i skonfigurowane w sposób zapewniający dostęp wszystkim pracownikom z danego piętra.

\paragraph{}
Każda z sal konferencyjnych została wyposażona w rzutnik multimedialny, a także komputer stacjonarny umożliwiający prowadzenie tele i wideokonferencji.
Ponadto w każdej z sal konferencyjnych umieszczony jest punkt dostępowy sieci bezprzewodowej.

\paragraph{}
Część parteru jednego z budynków została zaadaptowana jako serwerownia, w której umieszczono kilka serwerów. Serwery te pozwalają na przechowywanie repozytowiów kodu źródłowego, przprowadzanie testów oprogramowania, składownie i wymianę plików między pracownikami, kopie zapasowe danych, a także dostęp do baz danych wykorzystywanych do administracji oraz przy pracy nad projektami.

\paragraph{}
Systemy operacyjne dostępne dla pracowników:
\begin{itemize}
  \item Windows 7
  \item Ubuntu 11
  \item Mac OS Lion 10.7
\end{itemize}

\paragraph{}
Oprogramowanie wykorzystywane przez pracowników:
\begin{itemize}
  \item Komunikator internetowy (protokół XMPP)
  \item Program do tele i videokonferencji Skype
  \item Pogram pocztowy (dowolny)
  \item System kontroli wersji (svn, git)
  \item Oprogramowanie umożliwiające współdzielenie plików Samba
  \item Narzędzia służące do wytwarzania oprogramowania :
  \begin{itemize}
	\item Windows : Microsoft Visual Studio 2010, Eclipse
	\item Linux : Eclipse
	\item Mac OS : XCode
  \end{itemize}
  \item Program do pracy zdalnej TeamViewer
 \item Pakiet Office
\end{itemize}






