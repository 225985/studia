\section{Inwentaryzacja sprzętu i infrastruktury dostępnej w przedsiębiorstwie}

Z racji świadczonych usług dla klientów midzynarodowych niezbędne jest zapewnienie odpowiedniej przepustowości sieci do prowadzenia tele oraz videokonferencji.


\subsection{Telekonferencje}


\subsection{Videokonferencje}


\subsection{Wyposażenie}
\paragraph{}
Wyposażeniem każdego pracownika jest stacjonarny zestaw komputerowy, w skład którego wchodzą: jednostka centralna, mysz, klawiatura, monitor, kamera internetowa, słuchawki z mikrofonem.
Na każdym piętrze znajduje się sieciowe urządzenie wielofunkcyjne, podłączone i skonfigurowane w sposób zapewniający dostęp wszystkim pracownikom z danego piętra.

\paragraph{}
Każda z sal konferencyjnych została wyposażona w rzutnik multimedialny, a także komputer stacjonarny umożliwiający prowadzenie tele i wideokonferencji.
Ponadto w każdej z sal konferencyjnych umieszczony jest punkt dostępowy sieci bezprzewodowej.

\paragraph{}
Część parteru jednego z budynków została zaadaptowana jako serwerownia, w której umieszczono kilka serwerów. Serwery te pozwalają na przechowywanie repozytowiów kodu źródłowego, przprowadzanie testów oprogramowania, składownie i wymianę plików między pracownikami, kopie zapasowe danych, a także dostęp do baz danych wykorzystywanych do administracji oraz przy pracy nad projektami.

\paragraph{}
Systemy operacyjne dostępne dla pracowników:
\begin{itemize}
  \item Windows 7
  \item Ubuntu 11
  \item Mac OS Lion 10.7
\end{itemize}

\paragraph{}
Oprogramowanie wykorzystywane przez pracowników:
\begin{itemize}
  \item Komunikator internetowy (protokół XMPP)
  \item Program do tele i videokonferencji Skype
  \item Pogram pocztowy (dowolny)
  \item System kontroli wersji (svn, git)
  \item Oprogramowanie umożliwiające współdzielenie plików Samba
  \item Narzędzia służące do wytwarzania oprogramowania : 
  \begin{itemize}
	\item Windows : Microsoft Visual Studio 2010, Eclipse
	\item Linux : Eclipse
	\item Mac OS : XCode
  \end{itemize}
  \item Program do pracy zdalnej TeamViewer
 \item Pakiet Office
\end{itemize}




\subsection{Sieć bezprzewodowa}
\paragraph{}
W każdej sali konferencyjnej znajduje się punkt dostępowy sieci bezprzewodowej  oferujący jedynie dostęp do Internetu i innych komputerów w obrębie tej sali. Ma to na celu zwiększenie bezpiczeństwa i zablokowanie dostępu do sieci wewnętrznej firmy osobom postronnym. Sieć bezprzewodowa wykoanna będzie w standardzie 802.11g, będącym całkowicie zgodnym z poprzednim standardem 802.11b. Uwierzytelnienie użytkowników podłączających się do sieci odbywać się będzie za pomocą szyfrowania $WPA-PSK$.
\paragraph{}
Ze względu na charakter i wymagania pracy osób zajmujacych się produkcją oprogramowania dla urządzeń mobilnych, zachodzi potrzeba utworzenia bezpicznej sieci bezprzewodowej z dostępem do sieci wewnętrznej firmy. Sieć ta o ograniczonym zasięgu, dostępna będzie dla wybranych urządzeń o zautoryzowanych adresach $MAC$.

\subsection{Okablowanie}
\paragraph{}


\begin{itemize}
  \item Połączenie między dwoma budnykami firmy będzię zrealizowane za pomocą światłowodu 10 Gb/s

  \item Ze względu na fakt, iż główny ruch w sieci odbywa się między użytkownikiem, a serwerem, gdzie przechowywany jest kod i aplikacje testowe, połączenia pionowe powinny zapewniać większą przepustowość, niż połączenia  poziome. Ten typ połączeń wykonany zostanie za pomocą okablowania typu 1000Base-T Gigabit Ethernet, skrętka ekranowana kategori 6. 

  \item Okablowanie poziomie zostanie zrealizowane w technologi 100Base-T Fast Ethernet, skrętka foliowana UTP kategori 6. Decydujemy się na ten typ okablowania, ponieważ pojedynczy użytkownicy sieci, nie będą potrzebowali większej przepustowości niż oferowana przez ten typ połączenia

\end{itemize}

\subsection{VLAN}

\subsection{VPN}

