\newpage
\section{Projekt sieci}
\paragraph{}
Kolejnym etapem naszego projektu jest projekt logiczny sieci. Na Rysunku \textbf{TODO} przedstawiony zostało wzajemne położenie względem siebie budynków, które dla ulatwienia oznaczeń nazywać i opisu nazywać będziemy \textit{B1} i \textit{B2}. 

\subsection{Projekt logiczny sieci}
\paragraph{}
Ze względu na charakterystykę działania firmy i potrzeby odbiorcy, sieć podzielona została na sieci \textit{VLany}, odpowiadające każdemu zespołowi programistów. Pozwoli to na łatwe dołączanie osób do różnych zespołów (np. testerów oprogramowania), bez konieczności fizycznego przenoszenia komputea do pomieszczenia danego zespołu.

\paragraph{}
Na każdym piętrze dostepna będzie drukarka sieciowa, posiadająca adres z puli odpowedniego \textit{VLanu}.

\paragraph{}
W celu zapewnienia płynnego ruchu sieciowego na każdym piętrze znajdować sie będzie switch warstwy 2, podpiety do dwóch przełączników warstwy trzeciej znajdujących się an parterze każdego z budynków. Pomiędzy przełącznikami warstwy trzeciej skonfigurowane zostanie funkcja EtherChannel, pozwalający na połączeniu kilku ethernetowych łączy fizycznych w jedno logiczne. Dzięki temu, przełączniki mogą równomiernie rozkładać obciążenie na łączu, zapewnieniają wysokowydajnościowe połączenie pomiędzy urządzeniami sieciowymi.

\paragraph{} 
Przełączniki warstwy trzeciej połączone ze sobą zostaną dwunstożyłowym światłowodem w połączeniu każdy z każdym, by zapewnić niezawodność połączenia i zminimalizować ryzyko braku połączenia do internetu lub serwerowni \textit{B2} w wyniku awarii switch'a.

\paragraph{}
Dwa routery znjadujące się w \textit{B1} odpowiedzialne będą za zapewnienie niezawodnego połączenia z internetem. Pomiędzy ruoterami zastosowany zosatnie protokół VRRP, pozwalający na stworzenie klastra dostepowego, określanego jako wirtualny router. Przyśieszy to ruch sieciowy, oraz zwiększy bezpieczeństwo i niezawodność połączenia z internetem.  

\paragraph{}
Oznaczenia :
\begin{itemize}
	\item B\{X\}S\{Y\} - Switch warstwy 2, gdzie \{X\} - numer budynku, \{Y\} - numer piętra
	\item B\{X\}RS\{Y\} - Switch warstwy 3, gdzie \{X\} - numer budynku, \{Y\} - numer urządzenia
	\item R\{X\} - Router, gdzie \{X\} - numer urządzenia
	\item S\{X\} - Serwer, gdzie \{X\} - numer urządzenia
	\item B\{X\}P\{Y\}A\{Z\} - Access Point, gdzie \{X\} - numer budynku, \{Y\} - numer piętra, \{Z\} - numer urządzenie
	\item B\{X\}P\{Y\}K\{Z\} - Stacja robocza, gdzie \{X\} - numer budynku, \{Y\} - numer piętra, \{Z\} - numer urządzenie
	\item B\{X\}P\{Y\}D\{Z\} - Drukarka sieciowa, gdzie \{X\} - numer budynku, \{Y\} - numer piętra, \{Z\} - numer urządzenie
\end{itemize}

\paragraph{}
Podział na Vlany :
\begin{itemize}
	\item \textbf{TODO}
\end{itemize}

\newpage
\textbf{TODO legenda vlany}

\newpage
\textbf{TODO vlany}

\newpage
\textbf{TODO wstawic legende do rysunku bo cygan nie umial podzielic pliku}

\newpage
\textbf{TODO wstawic rysunek}

\newpage
\subsection{Konfiguracja adresacji \textit{IP}}
\paragraph{}
W celu zapewnienia odpowiedniej puli adresów, zapewniajacej możliwość skalowalności i robudowy sieci zdecydowaliśmy się na pulę adresów prywatnych klasy A, zaczynając od adresu sieci \textit{10.1.1.0}.
\paragraph{}
Poniżej przedstawiono pule adresowe dla poszczególnych \textit{VLanów} :
\begin{itemize}
	\item VLan WiFi : 10.1.1.0 - 10.1.1.255, Maska : 255.255.255.0
	\item VLan ZarzadIAdministracja : 10.1.2.0 - 10.1.2.255, Maska : 255.255.255.0
	\item VLan Team1 10.1.3.0 - 10.1.3.255, Maska : 255.255.255.0 
	\item VLan Team2 10.1.4.0 - 10.1.4.255, Maska : 255.255.255.0 
	\item VLan Team3 10.1.5.0 - 10.1.5.255, Maska : 255.255.255.0
	\item VLan Team4 10.1.6.0 - 10.1.6.255, Maska : 255.255.255.0  
	\item VLan Team5(Testerzy) 10.1.7.0 - 10.1.7.255, Maska : 255.255.255.0 
\end{itemize}


\subsection{Projekt fizyczny}
\paragraph{}
TODO

\subsection{Podłączenie do internetu}
\paragraph{}
TODO

\subsection{Bezpieczeństwo}
\paragraph{}
Projekt sieci powinien przewidywać zabezpieczenie jej przed następującymi czynnikami:
\begin{itemize}
	\item Ataki z zewnątrz :
	\begin{itemize}
		\item podsłuchanie ramek typu bradcast
		\item Ataki DoS
		\item Ataki MAC flooding
		\item Vlan leaking
	\end{itemize}
	\item Utrata danych
	\item Wirusy
	\item Czynniki fizyczne
	\begin{itemize}
		\item Uszkodzenia kabli
		\item Pożar
	\end{itemize}
\end{itemize}

\paragraph{}
TODO

\subsection{Kosztorys}
\paragraph{}
TODO