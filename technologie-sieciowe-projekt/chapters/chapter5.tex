\newpage
\section{Projekt sieci}
\paragraph{}
Kolejnym etapem naszego projektu jest projekt logiczny sieci. Na Rysunku \textbf{TODO} przedstawiony zostało wzajemne położenie względem siebie budynków, które dla ulatwienia oznaczeń nazywać i opisu nazywać będziemy \textit{B1} i \textit{B2}. 

\subsection{Projekt logiczny sieci}
\paragraph{}
Ze względu na charakterystykę działania firmy i potrzeby odbiorcy, sieć podzielona została na sieci \textit{VLany}, odpowiadające każdemu zespołowi programistów. Pozwoli to na łatwe dołączanie osób do różnych zespołów (np. testerów oprogramowania), bez konieczności fizycznego przenoszenia komputea do pomieszczenia danego zespołu.

\paragraph{}
Na każdym piętrze dostepna będzie drukarka sieciowa, posiadająca adres z puli odpowedniego \textit{VLanu}.

\paragraph{}
W celu zapewnienia płynnego ruchu sieciowego na każdym piętrze znajdować sie będzie switch warstwy 2, podpiety do dwóch przełączników warstwy trzeciej znajdujących się an parterze każdego z budynków. Pomiędzy przełącznikami warstwy trzeciej skonfigurowane zostanie funkcja EtherChannel, pozwalający na połączeniu kilku ethernetowych łączy fizycznych w jedno logiczne. Dzięki temu, przełączniki mogą równomiernie rozkładać obciążenie na łączu, zapewnieniają wysokowydajnościowe połączenie pomiędzy urządzeniami sieciowymi.

\paragraph{} 
Przełączniki warstwy trzeciej połączone ze sobą zostaną dwunstożyłowym światłowodem w połączeniu każdy z każdym, by zapewnić niezawodność połączenia i zminimalizować ryzyko braku połączenia do internetu lub serwerowni \textit{B2} w wyniku awarii switch'a.

\paragraph{}
Dwa routery znjadujące się w \textit{B1} odpowiedzialne będą za zapewnienie niezawodnego połączenia z internetem. Pomiędzy ruoterami zastosowany zosatnie protokół VRRP, pozwalający na stworzenie klastra dostępowego, określanego jako wirtualny router. Przyśieszy to ruch sieciowy, oraz zwiększy bezpieczeństwo i niezawodność połączenia z internetem.  

\paragraph{}
Oznaczenia :
\begin{itemize}
	\item B\{X\}S\{Y\} - Switch warstwy 2, gdzie \{X\} - numer budynku, \{Y\} - numer piętra
	\item B\{X\}RS\{Y\} - Switch warstwy 3, gdzie \{X\} - numer budynku, \{Y\} - numer urządzenia
	\item R\{X\} - Router, gdzie \{X\} - numer urządzenia
	\item S\{X\} - Serwer, gdzie \{X\} - numer urządzenia
	\item B\{X\}P\{Y\}A\{Z\} - Access Point, gdzie \{X\} - numer budynku, \{Y\} - numer piętra, \{Z\} - numer urządzenie
	\item B\{X\}P\{Y\}K\{Z\} - Stacja robocza, gdzie \{X\} - numer budynku, \{Y\} - numer piętra, \{Z\} - numer urządzenie
	\item B\{X\}P\{Y\}D\{Z\} - Drukarka sieciowa, gdzie \{X\} - numer budynku, \{Y\} - numer piętra, \{Z\} - numer urządzenie
\end{itemize}

\paragraph{}
Podział na Vlany :
\begin{itemize}
	\item \textbf{TODO}
\end{itemize}

\newpage
\textbf{TODO legenda vlany}

\newpage
\textbf{TODO vlany}

\newpage
\textbf{TODO wstawic legende do rysunku bo cygan nie umial podzielic pliku}

\newpage
\textbf{TODO wstawic rysunek}

\newpage
\subsection{Konfiguracja adresacji \textit{IP}}
\paragraph{}
W celu zapewnienia odpowiedniej puli adresów, zapewniajacej możliwość skalowalności i robudowy sieci zdecydowaliśmy się na pulę adresów prywatnych klasy A, zaczynając od adresu sieci \textit{10.1.1.0}.
\paragraph{}
Poniżej przedstawiono pule adresowe dla poszczególnych \textit{VLanów} :
\begin{itemize}
	\item VLan WiFi : 10.1.1.0 - 10.1.1.255, Maska : 255.255.255.0
	\item VLan ZarzadIAdministracja : 10.1.2.0 - 10.1.2.255, Maska : 255.255.255.0
	\item VLan Team1 10.1.3.0 - 10.1.3.255, Maska : 255.255.255.0 
	\item VLan Team2 10.1.4.0 - 10.1.4.255, Maska : 255.255.255.0 
	\item VLan Team3 10.1.5.0 - 10.1.5.255, Maska : 255.255.255.0
	\item VLan Team4 10.1.6.0 - 10.1.6.255, Maska : 255.255.255.0  
	\item VLan Team5(Testerzy) 10.1.7.0 - 10.1.7.255, Maska : 255.255.255.0 
	\item VLan UrzadzeniaMobilne 10.1.8.0 - 10.1.8.255, Maska : 255.255.255.0 
\end{itemize}


\subsection{Projekt fizyczny}
\subsubsection{Projekt okablowania}
\paragraph{}


\subsubsection{Tabela połączeń}
\paragraph{}
\begin{center}
    \begin{longtable}{|c|c|c|c|c|}
    \hline
    Gniazdo & Port Switch & Switch & Odległość \\ \hline
	1/0/1 & Fa0/1 & B1S1 & dupa \\ \hline
	1/0/2 & Fa0/2 & B1S1 & dupa \\ \hline
	1/0/3 & Fa0/3 & B1S1 & dupa \\ \hline
	1/0/4 & Fa0/4 & B1S1 & dupa \\ \hline
	1/0/5 & Fa0/5 & B1S1 & dupa \\ \hline
	1/0/6 & Fa0/6 & B1S1 & dupa \\ \hline
	1/0/7 & Fa0/7 & B1S1 & dupa \\ \hline
	1/0/8 & Fa0/8 & B1S1 & dupa \\ \hline
	1/0/9 & Fa0/9 & B1S1 & dupa \\ \hline
	1/0/10 & Fa0/10 & B1S1 & dupa \\ \hline
	1/0/11 & Fa0/11 & B1S1 & dupa \\ \hline
	1/0/12 & Fa0/12 & B1S1 & dupa \\ \hline
	1/0/13 & Fa0/13 & B1S1 & dupa \\ \hline
	1/0/14 & Fa0/14 & B1S1 & dupa \\ \hline
	1/0/15 & Fa0/15 & B1S1 & dupa \\ \hline
	1/0/16 & Fa0/16 & B1S1 & dupa \\ \hline
	1/0/17 & Fa0/17 & B1S1 & dupa \\ \hline
	1/0/18 & Fa0/18 & B1S1 & dupa \\ \hline
	1/0/19 & Fa0/19 & B1S1 & dupa \\ \hline
	1/0/20 & Fa0/20 & B1S1 & dupa \\ \hline
	1/0/21 & Fa0/21 & B1S1 & dupa \\ \hline
	1/0/22 & Fa0/22 & B1S1 & dupa \\ \hline
	1/0/23 & Fa0/23 & B1S1 & dupa \\ \hline
	1/0/24 & Fa0/24 & B1S1 & dupa \\ \hline
	1/0/25 & Fa0/25 & B1S1 & dupa \\ \hline
	1/0/26 & Fa0/26 & B1S1 & dupa \\ \hline
	1/0/27 & Fa0/27 & B1S1 & dupa \\ \hline
	1/0/28 & Fa0/28 & B1S1 & dupa \\ \hline
	1/0/29 & Fa0/29 & B1S1 & dupa \\ \hline
	1/0/30 & Fa0/30 & B1S1 & dupa \\ \hline
	1/0/31 & Fa0/31 & B1S1 & dupa \\ \hline
	1/0/32 & Fa0/32 & B1S1 & dupa \\ \hline
	1/0/33 & Fa0/33 & B1S1 & dupa \\ \hline
	1/0/34 & Fa0/34 & B1S1 & dupa \\ \hline
	1/0/35 & Fa0/35 & B1S1 & dupa \\ \hline
	1/0/36 & Fa0/36 & B1S1 & dupa \\ \hline
	1/0/37 & Fa0/37 & B1S1 & dupa \\ \hline
	1/0/38 & Fa0/38 & B1S1 & dupa \\ \hline
	1/0/39 & Fa0/39 & B1S1 & dupa \\ \hline
	1/0/40 & Fa0/40 & B1S1 & dupa \\ \hline
	1/0/41 & Fa0/41 & B1S1 & dupa \\ \hline
	1/0/42 & Fa0/42 & B1S1 & dupa \\ \hline
	1/0/43 & Fa0/43 & B1S1 & dupa \\ \hline
	1/0/44 & Fa0/44 & B1S1 & dupa \\ \hline
	1/0/45 & Fa0/45 & B1S1 & dupa \\ \hline
	1/0/46 & Fa0/46 & B1S1 & dupa \\ \hline
	1/0/47 & Fa0/47 & B1S1 & dupa \\ \hline
	1/0/48 & Fa0/48 & B1S1 & dupa \\ \hline

    \end{longtable}
\end{center}

\begin{center}
    \begin{longtable}{|c|c|c|c|c|}
    \hline
    Gniazdo & Port Switch & Switch & Odległość \\ \hline
	1/1/1 & Fa0/1 & B1S2 & dupa \\ \hline
	1/1/2 & Fa0/2 & B1S2 & dupa \\ \hline
	1/1/3 & Fa0/3 & B1S2 & dupa \\ \hline
	1/1/4 & Fa0/4 & B1S2 & dupa \\ \hline
	1/1/5 & Fa0/5 & B1S2 & dupa \\ \hline
	1/1/6 & Fa0/6 & B1S2 & dupa \\ \hline
	1/1/7 & Fa0/7 & B1S2 & dupa \\ \hline
	1/1/8 & Fa0/8 & B1S2 & dupa \\ \hline
	1/1/9 & Fa0/9 & B1S2 & dupa \\ \hline
	1/1/10 & Fa0/10 & B1S2 & dupa \\ \hline
	1/1/11 & Fa0/11 & B1S2 & dupa \\ \hline
	1/1/12 & Fa0/12 & B1S2 & dupa \\ \hline
	1/1/13 & Fa0/13 & B1S2 & dupa \\ \hline
	1/1/14 & Fa0/14 & B1S2 & dupa \\ \hline
	1/1/15 & Fa0/15 & B1S2 & dupa \\ \hline
	1/1/16 & Fa0/16 & B1S2 & dupa \\ \hline
	1/1/17 & Fa0/17 & B1S2 & dupa \\ \hline
	1/1/18 & Fa0/18 & B1S2 & dupa \\ \hline
	1/1/19 & Fa0/19 & B1S2 & dupa \\ \hline
	1/1/20 & Fa0/20 & B1S2 & dupa \\ \hline
	1/1/21 & Fa0/21 & B1S2 & dupa \\ \hline
	1/1/22 & Fa0/22 & B1S2 & dupa \\ \hline
	1/1/23 & Fa0/23 & B1S2 & dupa \\ \hline
	1/1/24 & Fa0/24 & B1S2 & dupa \\ \hline
	1/1/25 & Fa0/25 & B1S2 & dupa \\ \hline
	1/1/26 & Fa0/26 & B1S2 & dupa \\ \hline
	1/1/27 & Fa0/27 & B1S2 & dupa \\ \hline
	1/1/28 & Fa0/28 & B1S2 & dupa \\ \hline
	1/1/29 & Fa0/29 & B1S2 & dupa \\ \hline
	1/1/30 & Fa0/30 & B1S2 & dupa \\ \hline
	1/1/31 & Fa0/31 & B1S2 & dupa \\ \hline
	1/1/32 & Fa0/32 & B1S2 & dupa \\ \hline
	1/1/33 & Fa0/33 & B1S2 & dupa \\ \hline
	1/1/34 & Fa0/34 & B1S2 & dupa \\ \hline
	1/1/35 & Fa0/35 & B1S2 & dupa \\ \hline
	1/1/36 & Fa0/36 & B1S2 & dupa \\ \hline
	1/1/37 & Fa0/37 & B1S2 & dupa \\ \hline
	1/1/38 & Fa0/38 & B1S2 & dupa \\ \hline
	1/1/39 & Fa0/39 & B1S2 & dupa \\ \hline
	1/1/40 & Fa0/40 & B1S2 & dupa \\ \hline
	1/1/41 & Fa0/41 & B1S2 & dupa \\ \hline
	1/1/42 & Fa0/42 & B1S2 & dupa \\ \hline
	1/1/43 & Fa0/43 & B1S2 & dupa \\ \hline
	1/1/44 & Fa0/44 & B1S2 & dupa \\ \hline
	1/1/45 & Fa0/45 & B1S2 & dupa \\ \hline
	1/1/46 & Fa0/46 & B1S2 & dupa \\ \hline
	1/1/47 & Fa0/47 & B1S2 & dupa \\ \hline
	1/1/48 & Fa0/48 & B1S2 & dupa \\ \hline
    \end{longtable}
\end{center}

\begin{center}
    \begin{longtable}{|c|c|c|c|c|}
    \hline
    Gniazdo & Port Switch & Switch & Odległość \\ \hline
	1/2/1 & Fa0/1 & B1S3 & dupa \\ \hline
	1/2/2 & Fa0/2 & B1S3 & dupa \\ \hline
	1/2/3 & Fa0/3 & B1S3 & dupa \\ \hline
	1/2/4 & Fa0/4 & B1S3 & dupa \\ \hline
	1/2/5 & Fa0/5 & B1S3 & dupa \\ \hline
	1/2/6 & Fa0/6 & B1S3 & dupa \\ \hline
	1/2/7 & Fa0/7 & B1S3 & dupa \\ \hline
	1/2/8 & Fa0/8 & B1S3 & dupa \\ \hline
	1/2/9 & Fa0/9 & B1S3 & dupa \\ \hline
	1/2/10 & Fa0/10 & B1S3 & dupa \\ \hline
	1/2/11 & Fa0/11 & B1S3 & dupa \\ \hline
	1/2/12 & Fa0/12 & B1S3 & dupa \\ \hline
	1/2/13 & Fa0/13 & B1S3 & dupa \\ \hline
	1/2/14 & Fa0/14 & B1S3 & dupa \\ \hline
	1/2/15 & Fa0/15 & B1S3 & dupa \\ \hline
	1/2/16 & Fa0/16 & B1S3 & dupa \\ \hline
	1/2/17 & Fa0/17 & B1S3 & dupa \\ \hline
	1/2/18 & Fa0/18 & B1S3 & dupa \\ \hline
	1/2/19 & Fa0/19 & B1S3 & dupa \\ \hline
	1/2/20 & Fa0/20 & B1S3 & dupa \\ \hline
	1/2/21 & Fa0/21 & B1S3 & dupa \\ \hline
	1/2/22 & Fa0/22 & B1S3 & dupa \\ \hline
	1/2/23 & Fa0/23 & B1S3 & dupa \\ \hline
	1/2/24 & Fa0/24 & B1S3 & dupa \\ \hline
	1/2/25 & Fa0/25 & B1S3 & dupa \\ \hline
	1/2/26 & Fa0/26 & B1S3 & dupa \\ \hline
	1/2/27 & Fa0/27 & B1S3 & dupa \\ \hline
	1/2/28 & Fa0/28 & B1S3 & dupa \\ \hline
	1/2/29 & Fa0/29 & B1S3 & dupa \\ \hline
	1/2/30 & Fa0/30 & B1S3 & dupa \\ \hline
	1/2/31 & Fa0/31 & B1S3 & dupa \\ \hline
	1/2/32 & Fa0/32 & B1S3 & dupa \\ \hline
	1/2/33 & Fa0/33 & B1S3 & dupa \\ \hline
	1/2/34 & Fa0/34 & B1S3 & dupa \\ \hline
	1/2/35 & Fa0/35 & B1S3 & dupa \\ \hline
	1/2/36 & Fa0/36 & B1S3 & dupa \\ \hline
	1/2/37 & Fa0/37 & B1S3 & dupa \\ \hline
	1/2/38 & Fa0/38 & B1S3 & dupa \\ \hline
	1/2/39 & Fa0/39 & B1S3 & dupa \\ \hline
	1/2/40 & Fa0/40 & B1S3 & dupa \\ \hline
	1/2/41 & Fa0/41 & B1S3 & dupa \\ \hline
	1/2/42 & Fa0/42 & B1S3 & dupa \\ \hline
	1/2/43 & Fa0/43 & B1S3 & dupa \\ \hline
	1/2/44 & Fa0/44 & B1S3 & dupa \\ \hline
	1/2/45 & Fa0/45 & B1S3 & dupa \\ \hline
	1/2/46 & Fa0/46 & B1S3 & dupa \\ \hline
	1/2/47 & Fa0/47 & B1S3 & dupa \\ \hline
	1/2/48 & Fa0/48 & B1S3 & dupa \\ \hline

    \end{longtable}
\end{center}

\begin{center}
    \begin{longtable}{|c|c|c|c|c|}
    \hline
    Gniazdo & Port Switch & Switch & Odległość \\ \hline
	1/3/1 & Fa0/1 & B1S4 & dupa \\ \hline
	1/3/2 & Fa0/2 & B1S4 & dupa \\ \hline
	1/3/3 & Fa0/3 & B1S4 & dupa \\ \hline
	1/3/4 & Fa0/4 & B1S4 & dupa \\ \hline
	1/3/5 & Fa0/5 & B1S4 & dupa \\ \hline
	1/3/6 & Fa0/6 & B1S4 & dupa \\ \hline
	1/3/7 & Fa0/7 & B1S4 & dupa \\ \hline
	1/3/8 & Fa0/8 & B1S4 & dupa \\ \hline
	1/3/9 & Fa0/9 & B1S4 & dupa \\ \hline
	1/3/10 & Fa0/10 & B1S4 & dupa \\ \hline
	1/3/11 & Fa0/11 & B1S4 & dupa \\ \hline
	1/3/12 & Fa0/12 & B1S4 & dupa \\ \hline
	1/3/13 & Fa0/13 & B1S4 & dupa \\ \hline
	1/3/14 & Fa0/14 & B1S4 & dupa \\ \hline
	1/3/15 & Fa0/15 & B1S4 & dupa \\ \hline
	1/3/16 & Fa0/16 & B1S4 & dupa \\ \hline
	1/3/17 & Fa0/17 & B1S4 & dupa \\ \hline
	1/3/18 & Fa0/18 & B1S4 & dupa \\ \hline
	1/3/19 & Fa0/19 & B1S4 & dupa \\ \hline
	1/3/20 & Fa0/20 & B1S4 & dupa \\ \hline
	1/3/21 & Fa0/21 & B1S4 & dupa \\ \hline
	1/3/22 & Fa0/22 & B1S4 & dupa \\ \hline
	1/3/23 & Fa0/23 & B1S4 & dupa \\ \hline
	1/3/24 & Fa0/24 & B1S4 & dupa \\ \hline
	1/3/25 & Fa0/25 & B1S4 & dupa \\ \hline
	1/3/26 & Fa0/26 & B1S4 & dupa \\ \hline
	1/3/27 & Fa0/27 & B1S4 & dupa \\ \hline
	1/3/28 & Fa0/28 & B1S4 & dupa \\ \hline
	1/3/29 & Fa0/29 & B1S4 & dupa \\ \hline
	1/3/30 & Fa0/30 & B1S4 & dupa \\ \hline
	1/3/31 & Fa0/31 & B1S4 & dupa \\ \hline
	1/3/32 & Fa0/32 & B1S4 & dupa \\ \hline
	1/3/33 & Fa0/33 & B1S4 & dupa \\ \hline
	1/3/34 & Fa0/34 & B1S4 & dupa \\ \hline
	1/3/35 & Fa0/35 & B1S4 & dupa \\ \hline
	1/3/36 & Fa0/36 & B1S4 & dupa \\ \hline
	1/3/37 & Fa0/37 & B1S4 & dupa \\ \hline
	1/3/38 & Fa0/38 & B1S4 & dupa \\ \hline
	1/3/39 & Fa0/39 & B1S4 & dupa \\ \hline
	1/3/40 & Fa0/40 & B1S4 & dupa \\ \hline
	1/3/41 & Fa0/41 & B1S4 & dupa \\ \hline
	1/3/42 & Fa0/42 & B1S4 & dupa \\ \hline
	1/3/43 & Fa0/43 & B1S4 & dupa \\ \hline
	1/3/44 & Fa0/44 & B1S4 & dupa \\ \hline
	1/3/45 & Fa0/45 & B1S4 & dupa \\ \hline
	1/3/46 & Fa0/46 & B1S4 & dupa \\ \hline
	1/3/47 & Fa0/47 & B1S4 & dupa \\ \hline
	1/3/48 & Fa0/48 & B1S4 & dupa \\ \hline
    \end{longtable}
\end{center}

\begin{center}
    \begin{longtable}{|c|c|c|c|c|}
    \hline
    Gniazdo & Port Switch & Switch & Odległość \\ \hline
	2/0/1 & Fa0/1 & B2S1 & dupa \\ \hline
	2/0/2 & Fa0/2 & B2S1 & dupa \\ \hline
	2/0/3 & Fa0/3 & B2S1 & dupa \\ \hline
	2/0/4 & Fa0/4 & B2S1 & dupa \\ \hline
	2/0/5 & Fa0/5 & B2S1 & dupa \\ \hline
	2/0/6 & Fa0/6 & B2S1 & dupa \\ \hline
	2/0/7 & Fa0/7 & B2S1 & dupa \\ \hline
	2/0/8 & Fa0/8 & B2S1 & dupa \\ \hline
	2/0/9 & Fa0/9 & B2S1 & dupa \\ \hline
	2/0/10 & Fa0/10 & B2S1 & dupa \\ \hline
	2/0/11 & Fa0/11 & B2S1 & dupa \\ \hline
	2/0/12 & Fa0/12 & B2S1 & dupa \\ \hline
	2/0/13 & Fa0/13 & B2S1 & dupa \\ \hline
	2/0/14 & Fa0/14 & B2S1 & dupa \\ \hline
	2/0/15 & Fa0/15 & B2S1 & dupa \\ \hline
	2/0/16 & Fa0/16 & B2S1 & dupa \\ \hline
	2/0/17 & Fa0/17 & B2S1 & dupa \\ \hline
	2/0/18 & Fa0/18 & B2S1 & dupa \\ \hline
	2/0/19 & Fa0/19 & B2S1 & dupa \\ \hline
	2/0/20 & Fa0/20 & B2S1 & dupa \\ \hline
	2/0/21 & Fa0/21 & B2S1 & dupa \\ \hline
	2/0/22 & Fa0/22 & B2S1 & dupa \\ \hline
	2/0/23 & Fa0/23 & B2S1 & dupa \\ \hline
	2/0/24 & Fa0/24 & B2S1 & dupa \\ \hline
	2/0/25 & Fa0/25 & B2S1 & dupa \\ \hline
	2/0/26 & Fa0/26 & B2S1 & dupa \\ \hline
	2/0/27 & Fa0/27 & B2S1 & dupa \\ \hline
	2/0/28 & Fa0/28 & B2S1 & dupa \\ \hline
	2/0/29 & Fa0/29 & B2S1 & dupa \\ \hline
	2/0/30 & Fa0/30 & B2S1 & dupa \\ \hline
	2/0/31 & Fa0/31 & B2S1 & dupa \\ \hline
	2/0/32 & Fa0/32 & B2S1 & dupa \\ \hline
	2/0/33 & Fa0/33 & B2S1 & dupa \\ \hline
	2/0/34 & Fa0/34 & B2S1 & dupa \\ \hline
	2/0/35 & Fa0/35 & B2S1 & dupa \\ \hline
	2/0/36 & Fa0/36 & B2S1 & dupa \\ \hline
	2/0/37 & Fa0/37 & B2S1 & dupa \\ \hline
	2/0/38 & Fa0/38 & B2S1 & dupa \\ \hline
	2/0/39 & Fa0/39 & B2S1 & dupa \\ \hline
	2/0/40 & Fa0/40 & B2S1 & dupa \\ \hline
	2/0/41 & Fa0/41 & B2S1 & dupa \\ \hline
	2/0/42 & Fa0/42 & B2S1 & dupa \\ \hline
	2/0/43 & Fa0/43 & B2S1 & dupa \\ \hline
	2/0/44 & Fa0/44 & B2S1 & dupa \\ \hline
	2/0/45 & Fa0/45 & B2S1 & dupa \\ \hline
	2/0/46 & Fa0/46 & B2S1 & dupa \\ \hline
	2/0/47 & Fa0/47 & B2S1 & dupa \\ \hline
	2/0/48 & Fa0/48 & B2S1 & dupa \\ \hline
    \end{longtable}
\end{center}

\begin{center}
    \begin{longtable}{|c|c|c|c|c|}
    \hline
    Gniazdo & Port Switch & Switch & Odległość \\ \hline
	2/1/1 & Fa0/1 & B2S2 & dupa \\ \hline
	2/1/2 & Fa0/2 & B2S2 & dupa \\ \hline
	2/1/3 & Fa0/3 & B2S2 & dupa \\ \hline
	2/1/4 & Fa0/4 & B2S2 & dupa \\ \hline
	2/1/5 & Fa0/5 & B2S2 & dupa \\ \hline
	2/1/6 & Fa0/6 & B2S2 & dupa \\ \hline
	2/1/7 & Fa0/7 & B2S2 & dupa \\ \hline
	2/1/8 & Fa0/8 & B2S2 & dupa \\ \hline
	2/1/9 & Fa0/9 & B2S2 & dupa \\ \hline
	2/1/10 & Fa0/10 & B2S2 & dupa \\ \hline
	2/1/11 & Fa0/11 & B2S2 & dupa \\ \hline
	2/1/12 & Fa0/12 & B2S2 & dupa \\ \hline
	2/1/13 & Fa0/13 & B2S2 & dupa \\ \hline
	2/1/14 & Fa0/14 & B2S2 & dupa \\ \hline
	2/1/15 & Fa0/15 & B2S2 & dupa \\ \hline
	2/1/16 & Fa0/16 & B2S2 & dupa \\ \hline
	2/1/17 & Fa0/17 & B2S2 & dupa \\ \hline
	2/1/18 & Fa0/18 & B2S2 & dupa \\ \hline
	2/1/19 & Fa0/19 & B2S2 & dupa \\ \hline
	2/1/20 & Fa0/20 & B2S2 & dupa \\ \hline
	2/1/21 & Fa0/21 & B2S2 & dupa \\ \hline
	2/1/22 & Fa0/22 & B2S2 & dupa \\ \hline
	2/1/23 & Fa0/23 & B2S2 & dupa \\ \hline
	2/1/24 & Fa0/24 & B2S2 & dupa \\ \hline
	2/1/25 & Fa0/25 & B2S2 & dupa \\ \hline
	2/1/26 & Fa0/26 & B2S2 & dupa \\ \hline
	2/1/27 & Fa0/27 & B2S2 & dupa \\ \hline
	2/1/28 & Fa0/28 & B2S2 & dupa \\ \hline
	2/1/29 & Fa0/29 & B2S2 & dupa \\ \hline
	2/1/30 & Fa0/30 & B2S2 & dupa \\ \hline
	2/1/31 & Fa0/31 & B2S2 & dupa \\ \hline
	2/1/32 & Fa0/32 & B2S2 & dupa \\ \hline
	2/1/33 & Fa0/33 & B2S2 & dupa \\ \hline
	2/1/34 & Fa0/34 & B2S2 & dupa \\ \hline
	2/1/35 & Fa0/35 & B2S2 & dupa \\ \hline
	2/1/36 & Fa0/36 & B2S2 & dupa \\ \hline
	2/1/37 & Fa0/37 & B2S2 & dupa \\ \hline
	2/1/38 & Fa0/38 & B2S2 & dupa \\ \hline
	2/1/39 & Fa0/39 & B2S2 & dupa \\ \hline
	2/1/40 & Fa0/40 & B2S2 & dupa \\ \hline
	2/1/41 & Fa0/41 & B2S2 & dupa \\ \hline
	2/1/42 & Fa0/42 & B2S2 & dupa \\ \hline
	2/1/43 & Fa0/43 & B2S2 & dupa \\ \hline
	2/1/44 & Fa0/44 & B2S2 & dupa \\ \hline
	2/1/45 & Fa0/45 & B2S2 & dupa \\ \hline
	2/1/46 & Fa0/46 & B2S2 & dupa \\ \hline
	2/1/47 & Fa0/47 & B2S2 & dupa \\ \hline
	2/1/48 & Fa0/48 & B2S2 & dupa \\ \hline
    \end{longtable}
\end{center}

\begin{center}
    \begin{longtable}{|c|c|c|c|c|}
    \hline
    Gniazdo & Port Switch & Switch & Odległość \\ \hline
	2/2/1 & Fa0/1 & B2S3 & dupa \\ \hline
	2/2/2 & Fa0/2 & B2S3 & dupa \\ \hline
	2/2/3 & Fa0/3 & B2S3 & dupa \\ \hline
	2/2/4 & Fa0/4 & B2S3 & dupa \\ \hline
	2/2/5 & Fa0/5 & B2S3 & dupa \\ \hline
	2/2/6 & Fa0/6 & B2S3 & dupa \\ \hline
	2/2/7 & Fa0/7 & B2S3 & dupa \\ \hline
	2/2/8 & Fa0/8 & B2S3 & dupa \\ \hline
	2/2/9 & Fa0/9 & B2S3 & dupa \\ \hline
	2/2/10 & Fa0/10 & B2S3 & dupa \\ \hline
	2/2/11 & Fa0/11 & B2S3 & dupa \\ \hline
	2/2/12 & Fa0/12 & B2S3 & dupa \\ \hline
	2/2/13 & Fa0/13 & B2S3 & dupa \\ \hline
	2/2/14 & Fa0/14 & B2S3 & dupa \\ \hline
	2/2/15 & Fa0/15 & B2S3 & dupa \\ \hline
	2/2/16 & Fa0/16 & B2S3 & dupa \\ \hline
	2/2/17 & Fa0/17 & B2S3 & dupa \\ \hline
	2/2/18 & Fa0/18 & B2S3 & dupa \\ \hline
	2/2/19 & Fa0/19 & B2S3 & dupa \\ \hline
	2/2/20 & Fa0/20 & B2S3 & dupa \\ \hline
	2/2/21 & Fa0/21 & B2S3 & dupa \\ \hline
	2/2/22 & Fa0/22 & B2S3 & dupa \\ \hline
	2/2/23 & Fa0/23 & B2S3 & dupa \\ \hline
	2/2/24 & Fa0/24 & B2S3 & dupa \\ \hline
	2/2/25 & Fa0/25 & B2S3 & dupa \\ \hline
	2/2/26 & Fa0/26 & B2S3 & dupa \\ \hline
	2/2/27 & Fa0/27 & B2S3 & dupa \\ \hline
	2/2/28 & Fa0/28 & B2S3 & dupa \\ \hline
	2/2/29 & Fa0/29 & B2S3 & dupa \\ \hline
	2/2/30 & Fa0/30 & B2S3 & dupa \\ \hline
	2/2/31 & Fa0/31 & B2S3 & dupa \\ \hline
	2/2/32 & Fa0/32 & B2S3 & dupa \\ \hline
	2/2/33 & Fa0/33 & B2S3 & dupa \\ \hline
	2/2/34 & Fa0/34 & B2S3 & dupa \\ \hline
	2/2/35 & Fa0/35 & B2S3 & dupa \\ \hline
	2/2/36 & Fa0/36 & B2S3 & dupa \\ \hline
	2/2/37 & Fa0/37 & B2S3 & dupa \\ \hline
	2/2/38 & Fa0/38 & B2S3 & dupa \\ \hline
	2/2/39 & Fa0/39 & B2S3 & dupa \\ \hline
	2/2/40 & Fa0/40 & B2S3 & dupa \\ \hline
	2/2/41 & Fa0/41 & B2S3 & dupa \\ \hline
	2/2/42 & Fa0/42 & B2S3 & dupa \\ \hline
	2/2/43 & Fa0/43 & B2S3 & dupa \\ \hline
	2/2/44 & Fa0/44 & B2S3 & dupa \\ \hline
	2/2/45 & Fa0/45 & B2S3 & dupa \\ \hline
	2/2/46 & Fa0/46 & B2S3 & dupa \\ \hline
	2/2/47 & Fa0/47 & B2S3 & dupa \\ \hline
	2/2/48 & Fa0/48 & B2S3 & dupa \\ \hline
    \end{longtable}
\end{center}

\subsection{Podłączenie do internetu}
\paragraph{}
TODO

\subsection{Bezpieczeństwo}
\paragraph{}
Projekt sieci powinien przewidywać zabezpieczenie jej przed następującymi czynnikami:
\begin{itemize}
	\item Ataki z zewnątrz :
	\begin{itemize}
		\item Podsłuchanie ramek typu bradcast
		\item Ataki DoS
		\item Ataki MAC flooding
		\item Vlan leaking
	\end{itemize}
	\item Utrata danych
	\item Wirusy
	\item Czynniki fizyczne
	\begin{itemize}
		\item Uszkodzenia kabli
		\item Pożar
	\end{itemize}
\end{itemize}

\paragraph{}
By ochronić stacje robocze przed wirusami, zalecane przez nas jest komercyjne oprogramowanie \textit{NOD32}, posiadające odpowiednie wersje zarówno pod system Windows jak i Linux. W skład pakietu wchodza również aplikacje SpyWare i RootKit. \textit{NOD32} umożliwia również blokowanie niebezpiecznych aplikacji i witryn internetowych.

\paragraph{}
Projektowaną sięć należy zabezpieczyć przed czynnikami fizycznymi jak uszkodzenia kabli, przegrzanie serwerów (ewentualne pożary). W celu uniknięcia mechanicznych uszkodzeń okablowania (czynnik ludzki, gryzonie), przewody prowadzone są w korytkach nasciennych, listwach przypodłogowych, używane są gniazda RJ-45 podtynkowe. W serwerowni, gdzie przez ciągłą pracę serwerów, temperatura otoczenia może gwałtownie wzrosnąć, zalecane jest założenie klimatyzacji oraz umieszczenie gaśnicy proszkowej, odpowiedniej do gaszenia pożarów w instalacjach elektrycznych. Takie gaśnice powinny się również znaleźć na każdym piętrze budynków zgodnie z wymogami BHP.

\paragraph{}
Ze względu na charakterystykę oferowanych usług przez firmę zabezpieczenie kodu przechowywanego na serwerach systemu kontroli wersji, a także danych przechowywanych w bazie danych jest jednym z najważniejszych czynników. Obowiązkowe jest tworzenie kopii zapasowych codziennie w nocy, by nie obciążać zbytnio ruchu sieciowego. By chronić dane przed atakami z zewnątrz dostęp do nich zagwarantowany zsoatnie tylko iwyłącznie z sieci zewnętrznej.

\paragraph{}
Access Pointy, gawarantujące dostęp do internetu uzytkownikom z zewnątrz (np. klientą), zawierać się będą w jednej sieci VLan, nie mającej dostepu do zasobów wewnętrznych.

\subsection{Kosztorys}
\paragraph{}
TODO