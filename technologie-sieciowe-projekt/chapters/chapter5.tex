\newpage
\section{Projekt sieci}
\paragraph{}
Kolejnym etapem naszego projektu jest projekt logiczny sieci. Na Rysunku \textbf{TODO} przedstawiony zostało wzajemne położenie względem siebie budynków, które dla ulatwienia oznaczeń nazywać i opisu nazywać będziemy \textit{B1} i \textit{B2}. 

\subsection{Projekt logiczny sieci}
\paragraph{}
Ze względu na charakterystykę działania firmy i potrzeby odbiorcy, sieć podzielona została na na \textit{VLany}, odpowiadające odpowiednio każdemu zespołowi programistów. Pozwoli to na łatwe dołączanie osób do różnych teamó'w (np. testerów orpogramowania), bez konieczności fizycznego przenoszenia komputea do pomieszczenia danego pomieszczenia.

\paragraph{}
Na każdym piętrze dostepna będzie drukarka sieciowa, posiadająca adres z puli odpowedniego \textit{VLanu}.

\paragraph{}
W celu zapewnienia płynnego ruchu sieciowego na każdym piętrze znajdować sie będzie switch warstwy 2, podpiety do dwóch przełączników warstwy trzeciej znajdujących się an parterze każdego z budynków. Pomiędzy przełącznikami warstwy trzeciej skonfigurowane zostanie EtherChanel, pozwalający na \textbf{TODO}. 

\paragraph{} 
Przełączniki warstwy trzeciej połączone ze sobą zostaną dwunstożyowym światłowodem w połączeniu każy z każdym, by zapewnić niezawodność połącenia i zminimalizować ryzyko braku połączenia do internetu lub serwerowni \textit{B2} w wyniku awarii switch'a.

\paragraph{}
Dwa routery znjadujące się w \textit{B1} odpowiedzialne będą za zapewnienie niezawodnego połączenia z internetem. Pomiędzy ruoterami zastosowany zosatnie protokół \textbf{TODO} by \textbf{TODO}

\paragraph{}
Oznaczenia :
\begin{itemize}
	\item \textbf{TODO}
\end{itemize}

\paragraph{}
Podział na Vlany :
\begin{itemize}
	\item \textbf{TODO}
\end{itemize}

\newpage
\textbf{TODO legenda vlany}

\newpage
\textbf{TODO vlany}

\newpage
\textbf{TODO wstawic legende do rysunku bo cygan nie umial podzielic pliku}

\newpage
\textbf{TODO wstawic rysunek}

\newpage
\subsection{Konfiguracja adresacji \textit{IP}}
\paragraph{}
W celu zapewnienia odpowiedniej puli adresów, zapewniajacej możliwość skalowalności i robudowy sieci zdecydowaliśmy się na pulę adresów prywatnych klasy A, zaczynając od adresu sieci \textit{10.1.1.0}.
\paragraph{}
Poniżej przedstawiono pule adresowe dla poszczególnych \textit{VLanów} :
\begin{itemize}
	\item TODO
\end{itemize}


\subsection{Projekt fizyczny}
\paragraph{}
TODO

\subsection{Podłączenie do internetu}
\paragraph{}
TODO

\subsection{Bezpieczeństwo}
\paragraph{}
TODO

\subsection{Kosztorys}
\paragraph{}
TODO