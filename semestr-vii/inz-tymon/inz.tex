\documentclass[wide,a4paper,titlepage,12pt] {article}
\usepackage{polski}
\usepackage[utf8]{inputenc}
\usepackage{listings}
\usepackage{slashbox}
\usepackage[table]{xcolor}
\usepackage{graphicx,pdflscape}
\usepackage{placeins}


\title{Projekt inżynierski}
\author{Tymon Tobolski (181037)}

\makeatletter
\renewcommand{\maketitle}{
\begin{titlepage}
  \begin{center}
    \vspace*{3cm}
    \LARGE \@title \par
    \vspace{2cm}
    \textit{\small Autor:}\par
    \normalsize \@author\par \normalsize
    \vspace{3cm}
    \textit{\small Prowadzący:}\par
    dr inż. Henryk Maciejewski \par
    \vspace{2cm}
    Wydział Elektroniki\\ %III rok\\ Pn TP 08.15 - 11.00\par
    % \vspace{4cm}
    % \small 7 listopada 2011
  \end{center}
\end{titlepage}
}
\begin{document}
\maketitle
  \section{Cel projektu}
  \paragraph{}
  Celem projektu jest stworzenie serwisu umożliwiającego możliwie
  jak najprostszą publikację projektów open-source w języku Scala.


  Scala jest stosunkowo młodym językiem działającym na platformie JVM,
  który zdobywa coraz większą popularność zarówno w środowiskach
  akademickich/badawczych jak i w dużych korporacjach.

  Ze względu na duży wpływ środowiska języka Java biblioteki języka Scala
  są dystrybuowane jako pakiety .jar. Publikacja biblioteki najczęściej sprowadza
  się do utrzymania własnego serwera, a korzystanie z takich bliblioteki wymaga
  znania dokładnego położenia danej biblioteki. Dla porównania, w środowiskach innych
  języków taki jak Ruby\footnote{http://rubygems.org} czy Clojure\footnote{http://clojars.org} istnieje jedno centralne miejsce dla
  bibliotek open-source.

  \section{Założenia projektowe}
  \paragraph{}
  Podstawowe
  \begin{itemize}
      \item Rejestracja użytkowników przy użyciu protokołu OAuth z serwisu Github\footnote{http://github.com}
      \item Katalog bibliotek wraz z ich szczegółami (opis, słowa kluczowe, odnośniki do źródeł, dokumentacji, itp.)
      \item Plugin do sbt (simple-build-tool)\footnote{https://github.com/harrah/xsbt/wiki} pozwalający na publikacje, wyszukiwanie oraz instalacje bibliotek
  \end{itemize}

  \paragraph{}
  Opcjonalne
  \begin{itemize}
      \item JSON API dla katalogu
      \item Integracja z Github Hooks\footnote{http://help.github.com/post-receive-hooks/}
  \end{itemize}


\end{document}
