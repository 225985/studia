\pdfoutput=1
\pdfcompresslevel=9
\pdfinfo
{
    /Author ()
    /Title ()
    /Subject ()
    /Keywords ()
}

%\newcommand*{\memfontfamily}{pnc}
%\newcommand*{\memfontpack}{newcent}
\documentclass[a4paper,onecolumn,oneside,12pt]{memoir}
% Wydruk do archiwum
%\documentclass[a4paper,onecolumn,twoside,10pt]{memoir} 
%\renewcommand{\normalsize}{\fontsize{8pt}{10pt}\selectfont}

\usepackage[utf8]{inputenc}
\usepackage[T1]{fontenc}
\usepackage[polish]{babel}
\usepackage{setspace}
\usepackage{tabularx}
\usepackage{color,calc}
%\usepackage{soul} % pakiet z komendami do podkreœlania tekstu
%\usepackage{fourier} % pakiet zmieniaj¹cy czcionki na utopia
\usepackage{times} % pakiet zmieniaj¹cy czcionki na times 
\usepackage{pdflscape}
%\usepackage{longtable}
%\usepackage{ltxtable}
%\usepackage{tabulary}

%%%%%% Ustawienia odpowiedzialne za sposób ³amania dokumentu
%\hyphenpenalty=10000		% nie dziel wyrazów zbyt czêsto
\clubpenalty=10000      %kara za sierotki
\widowpenalty=10000  % nie pozostawiaj wdów
\brokenpenalty=10000		% nie dziel wyrazów miêdzy stronami
\exhyphenpenalty=999999		% nie dziel s³ów z myœlnikiem
\righthyphenmin=3			% dziel minimum 3 litery

%\tolerance=4500
%\pretolerance=250
%\hfuzz=1.5pt
%\hbadness=1450

%ustawienia rozmiarów: tekstu, stopki, marginesów 
\setlength{\textwidth}{\paperwidth}
\addtolength{\textwidth}{-5cm}
\setlength{\textheight}{\paperheight}
\addtolength{\textheight}{-5cm}
\setlength{\oddsidemargin}{-0.04cm} % domyœlnie jest 1 cal = 2.54 cm, st¹d -0.04 da margines 2.5cm
\setlength{\evensidemargin}{-0.04cm} % domyœlnie jest 1 cal = 2.54 cm, st¹d -0.04 da margines 2.5cm
\topmargin -1.25cm
\footskip 1.4cm 

\linespread{1.3}

\usepackage{ifpdf}
%\newif\ifpdf \ifx\pdfoutput\undefined
%\pdffalse % we are not running PDFLaTeX
%\else
%\pdfoutput=1 % we are running PDFLaTeX
%\pdftrue \fi
\ifpdf
\usepackage[pdftex]{graphicx,hyperref}
\DeclareGraphicsExtensions{.pdf,.jpg,.mps,.png}
\pdfcompresslevel=9
\else
\usepackage{graphicx}
\DeclareGraphicsExtensions{.eps,.ps,.jpg,.mps,.png}
\fi
\sloppy

%\graphicspath{{rys01/}{rys02/}}

\renewcommand{\topfraction}{0.95}
\renewcommand{\bottomfraction}{0.95}
\renewcommand{\textfraction}{0.05}
\renewcommand{\floatpagefraction}{0.35}


%%%%%%%%%%%%%%%%%%%%%%%%%%%%%%%%%%%%%%%
%                  Definicja strony tytu³owej 
%%%%%%%%%%%%%%%%%%%%%%%%%%%%%%%%%%%%%%%
\makeatletter
%Uczelnia
\newcommand\uczelnia[1]{\renewcommand\@uczelnia{#1}}
\newcommand\@uczelnia{}
%Wydzia³
\newcommand\wydzial[1]{\renewcommand\@wydzial{#1}}
\newcommand\@wydzial{}
%Kierunek
\newcommand\kierunek[1]{\renewcommand\@kierunek{#1}}
\newcommand\@kierunek{}
%SpecjalnoϾ
\newcommand\specjalnosc[1]{\renewcommand\@specjalnosc{#1}}
\newcommand\@specjalnosc{}
%Tytu³ po angielsku
\newcommand\titleEN[1]{\renewcommand\@titleEN{#1}}
\newcommand\@titleEN{}
%Tytu³ krótki
\newcommand\titleShort[1]{\renewcommand\@titleShort{#1}}
\newcommand\@titleShort{}
%Promotor
\newcommand\promotor[1]{\renewcommand\@promotor{#1}}
\newcommand\@promotor{}

\def\maketitle{%
  \null
  \pagestyle{empty}%
	{\centering\vspace{-1cm}
		{\fontsize{22pt}{24pt}\selectfont \@uczelnia}\\[0.4cm]
		{\fontsize{22pt}{24pt}\selectfont \@wydzial }\\[0.5cm]
		\hrule \vspace*{0.7cm}
	}
{\flushleft\fontsize{14pt}{16pt}\selectfont%
\begin{tabular}{ll}
Kierunek: & \@kierunek\\
Specjalność & \@specjalnosc\\
\end{tabular}\\[1.3cm]
}
{\centering
%{\fontsize{24pt}{26pt}\selectfont PRACA DYPLOMOWA}\\[0.5cm]
%{\fontsize{24pt}{26pt}\selectfont MAGISTERSKA}\\[2cm]
{\fontsize{24pt}{26pt}\selectfont Projekt Inżynierski}\\[1.5cm]
}
%
\begin{tabularx}{\linewidth}{p{6cm}>{\centering\arraybackslash}X}
		&{\fontsize{16pt}{18pt}\selectfont \@title}\\[5mm] 	%UWAGA: tutaj jest miejsce na tyty³ w jêzyku polskim
		&{\fontsize{16pt}{18pt}\selectfont \@titleEN}\\[10mm] %UWAGA: tutaj jest miejsce na tyty³ w jêzyku angielskim
\end{tabularx}
\vfill
\begin{tabularx}{\linewidth}{p{6cm}l}
		%UWAGA: tutaj jest miejsce na autora pracy
		&{\fontsize{16pt}{18pt}\selectfont Autor:}\\[5mm]
		&{\fontsize{14pt}{16pt}\selectfont \@author}\\[10mm]
		%UWAGA: tutaj jest miejsce na promotora pracy
		&{\fontsize{16pt}{18pt}\selectfont Prowadzący pracę:}\\[5mm]
		&{\fontsize{14pt}{16pt}\selectfont \@promotor}\\[10mm]
		&{\fontsize{16pt}{18pt}\selectfont Ocena pracy:}\\[20mm]
	\end{tabularx}
\hrule\vspace*{0.3cm}
{\centering
%{\fontsize{24pt}{26pt}\selectfont PRACA DYPLOMOWA}\\[0.5cm]
%{\fontsize{24pt}{26pt}\selectfont MAGISTERSKA}\\[2cm]
{\fontsize{16pt}{18pt}\selectfont \@date}\\[0cm]
}
\normalfont
 \cleardoublepage
}
\makeatother
%%%%%%%%%%%%%%%%%%%%%%%%%%%%%%%%%%%%%%%
%                  Styl rozdzia³ów 
%%%%%%%%%%%%%%%%%%%%%%%%%%%%%%%%%%%%%%%
\setcounter{secnumdepth}{3}
\setcounter{tocdepth}{3}
%\definecolor{niceblue}{rgb}{.168,.234,.671}

%\AtBeginDocument{% 
        \addto\captionspolish{% 
        \renewcommand{\tablename}{Tab.}% 
}%} 

%\AtBeginDocument{% 
%        \addto\captionspolish{% 
%        \renewcommand{\chaptername}{Rozdzia³}% 
%}} 

%\AtBeginDocument{% 
        \addto\captionspolish{% 
        \renewcommand{\figurename}{Rys.}% 
}%}
        \addto\captionspolish{% 
        \renewcommand{\bibname}{Literatura}% 
}

%%%%%%%%%%%%%%%%%%%%%%%%%%%%%%%%%%%%%%%%%%%%%%%%%%%%%%%%%%%%%%%%%%                  Styl wyliczenia (opis skrótów) 
%%%%%%%%%%%%%%%%%%%%%%%%%%%%%%%%%%%%%%%%%%%%%%%%%%%%%%%%%%%%%%%%%
\newenvironment{Ventry}[1]%
 {\begin{list}{}{\renewcommand{\makelabel}[1]{\textbf{##1}\hfill}%
   \settowidth{\labelwidth}{\textbf{#1}}%
   \setlength{\leftmargin}{3cm}}}%
 {\end{list}}

\addtopsmarks{headings}{%
\nouppercaseheads % added at the beginning
}{%
\createmark{chapter}{both}{shownumber}{}{. \space}
%\createmark{chapter}{left}{shownumber}{}{. \space}
\createmark{section}{right}{shownumber}{}{. \space}
}%use the new settings
\pagestyle{headings}

\newlength\mytemplengtha

\setcounter{secnumdepth}{2}
\setcounter{tocdepth}{2}
\setsecnumdepth{subsection} % activating subsubsec numbering in doc

\makeatletter
\copypagestyle{outer}{headings}
\makeoddhead{outer}{}{}{\slshape\rightmark}
\makeevenhead{outer}{\slshape\leftmark}{}{}
\makeoddfoot{outer}{\@author:~\@titleShort}{}{\thepage}
\makeevenfoot{outer}{\thepage}{}{\@author:~\@title}
\makeheadrule{outer}{\linewidth}{\normalrulethickness}
\makefootrule{outer}{\linewidth}{\normalrulethickness}{6pt}
\makeatother

% fix plain
%\copypagestyle{plain}{outer} % overwrite plain with outer
\makeoddhead{plain}{}{}{} % remove right header
\makeevenhead{plain}{}{}{} % remove left header
\makeevenfoot{plain}{}{}{}
\makeoddfoot{plain}{}{}{}

%\copypagestyle{plain}{outer} % overwrite plain with outer
\makeoddhead{empty}{}{}{} % remove right header
\makeevenhead{empty}{}{}{} % remove left header
\makeevenfoot{empty}{}{}{}
\makeoddfoot{empty}{}{}{}

\pagestyle{outer}


\makeatletter
\def\@seccntformat#1{\csname the#1\endcsname.\quad}
\def\numberline#1{\hb@xt@\@tempdima{#1\if&#1&\else.\fi\hfil}}
\makeatother

\renewcommand{\chapternumberline}[1]{#1.\quad}
\renewcommand{\cftchapterdotsep}{\cftdotsep}

\begin{document}
\title{Internetowy system wspomagania treningów    aerobowych z aplikacją mobilną}
\titleShort{Internetowy system wspomagania ...}
\titleEN{Internet system for aerobic training with a mobile application}
\author{Jacek Wieczorek}
\uczelnia{Politechnika Wrocławska}
\wydzial{Wydział Elektroniki}
\kierunek{Informatyka}
\specjalnosc{Inżynieria Internetowa}
\promotor{dr inż. Tomasz Walkowiak}
\date{Wrocław, 2012}
\maketitle

\pagestyle{outer}

\tableofcontents


\chapter{Wprowadzenie}

\section{Cel pracy}
\paragraph{}
Celem niniejszego projektu inżynierskiego jest zaprojektowanie oraz implementacja systemu do wspomagania treningów areobowych. Główną częścią systemu jest aplikacja internetowa wraz z aplikacją mobilną.



\section{Istniejące rozwiązania}
\paragraph{}
Na rynku dostępne jest wiele rozwiązań pozwalających na zarządzanie przebiegami treningów. Jednak w większość są to rozwiązania komercyjne, 


\chapter{Komponenty wykorzystywane w projekcie}

\section{ C\# i .NET }
\paragraph{}
C\# jest obiektowym językiem programowania stworzonym dla firmy Microsoft przez zaspół Andersa Hejslberga. Język ten powstał jako alternatywa dla Javy. Programy napisane w C\# kompilowane są do natywnego języka Common Intermediate Language (CLI), czyli kodu pośredniego w środowisku takim jak .NET czy Mono.
\paragraph{}
.NET Framework jest platformą programistyczną obejmującą środowisko uruchomieniowe CLR (Common Language Runtime) oraz biblioteki klas z podstawowymi funkcjonalnościami. Zadaniem .NET jest zarządzanie elementami systemu, takimi jak : kod aplikacji, pamięć oraz zabezpieczenia. .NET umożliwia uruchamianie aplikacji zarówno na systemie, w którym istnieje działająca implementacja platformy, a także po stronie serwera IIS. Platforma .NET nie określa jednoznacznie języka programowania. Aplikacje działające na niej mogą być napisane w C\#, F\#, J\#, VB.NET, C++/CLI czy Delphi 8. 

\section{Framework ASP.NET MVC 3}

\paragraph{}
ASP.NET MVC 3 jest platformą do budowy aplikacji internetowych korzystającą z wzorca MVC (Model - Widok - Kontroler) bazującej na platformie ASP.NET. Zaletą z korzystania ze wzorca MVC jest odseparowanie warstw aplikacji i logiki biznesowej. Aplikacje stworzone za pomocą tego frameworka są z reguły łatwiejsze w rozbudowie i testowaniu (testy jednostkowe).

\paragraph{}
ASP.NET MVC bazuje na tradycyjnym silniku ASP.NET, dzięki czemu można wykorzystać wiele mechanizmów stworzonych dla tej platformy jak zarządzanie cachem, autoryzacja czy monitorowanie. Mechanizm mapowania adresów umożliwia łatwą budowę aplikacji w oparciu o architekturę REST. Model programistyczny silnie bazuje na interfejsach, co pozwala na łatwą rozbudowę i testowanie komponentów.

\section{ Windows Communication Foundation }
\paragraph{}
Windows Communication Foundation (w skrócie WCF) jest platformą służącą do budowy aplikacji zorientowanych serwisowo. Pozwala na budowę aplikacji, które mogą działać na serwerze IIS lub jako część systemu. WCF pozwala na komunikację między platformową za pomocą technologii SOAP prz użyciu prostych formatów XML lub JSON. Kluczowym aspektem biznesowym platformy WCF jest zapewnienie wysokiej wydajności, przy równie wysokiej niezawodności działania systemu.

\section{Baza daynych}

\subsection{Microsoft SQL Server 2012}
\paragraph{}
Microsoft SQL Server (MS SQL) to system zarządzania relacyjnymi bazami danych. Jako język zapytań używany jest Transact-SQL będący rozwinięciem standardu ANSi/ISO. MS SQL jest platformą bazodanową typu klient-serwer charakteryzującą się wysoką wydajnością, niezawodnością i bezpieczeństwem.

\subsection{SQLite} % (fold)
\paragraph{} % (fold)
SQLite jest systemem zarządzania bazą danych i biblioteką języka C, implementującą taki system. Charakteryzuje się przetrzymywaniem danych w jednym pliku (do 2 TB). Baza utrzymywana jest przy użyciu B-drzewa. Bazy danych zapisywane są jako pliki binarne. SQLite jest szeroko wykorzystywany w systemach wbudowanych jak i przez platformę Android.

\section{Windows Azure} % (fold)

\paragraph{} % (fold)
Windows Azure jest platwormą chmurową stworzoną przez firmę Microsoft. Udostępnia ona mechanizmy do przetwarzania danych (Windows Azure Compute) oraz do ich przechowywania (Windows Azure Storage  i SQL Azure).
\paragraph{} % (fold)
\label{par:}

% paragraph  (end)
Windows Azure pozwala na budowanie aplikacji we wsztskich technologiach, które można wykorzystać na zwykłym systemie Windows. Poza C\# i innymi językami platformy .NET można tworzyć oprogramowanie w takich językach jak Java, C++, PHP czy Python. Windows Azure zapewnia integracją z popularnymi środowiskami programistycznymi jak Microsoft Visual Studio czy Eclipse.
\paragraph{} % (fold)
\label{par:}

% paragraph  (end)
W celu ułatwienia procesu wytwarzania oprogramowania po tę platformę, firma Microsoft udostępniła dedykowane dla różnych języków zestawy narządzi programistycznych. Ważną ich częścią jest emulator chmury, dzięki któremu można w wygodny sposób testować działąnie aplikacji, bez konieczności instalacji oprogramowania w chmurze.

\section{Android SDK} % (fold)
\label{sec:android_sdk}

\paragraph{} % (fold)
\label{par:}

% paragraph  (end)
Android SDK jest paczką narzędzi programistycznych, niezbędnych w procesie tworzenia i testowania aplikacji na platformę Android. Ważnymi elementami Android SDK są wtyczki dla środowiska programistycznego Eclipse, symulator systemu.

\section{CoffeeScript} % (fold)
\label{sec:coffeescript}

% section coffeescript (end)
CoffeeScript jest językiem skryptowym kompilowalnym do kodu JavaScript. Główną zaletą tego języka jest wyeliminowanie nawiasów, klamr i średników, co zmniejsza ryzyko popełnienia składniowych błędów, uniemożliwiających poprawne działanie aplikacji internetowej.

\section{HTML5 i CSS3} % (fold)
\label{par:html5_i_css3}

% paragraph html5_i_css3 (end)
HTML5 jest językiem wykorzystywanym do tworzenia stron WWW. Jest rozwinięciem standardu języka HTML4, posiadając wiele udogodnień dla programistów. Główne technologie wersji 5 jest obsługa grafiki 2D\/3D, audio i video oraz pełne wsparcie dla kaskadowych arkuszów styli w wersji 3 (CSS3). Dzięki temu, nie trzeba posiłkować się bibliotekami JavaScript oraz technologią Flash, by osiągnąć oczekiwane efekty wizualnei użytkowe aplikacji.


\section{Inne biblioteki wykorzystane w projekcie} % (fold)
\label{sec:inne_biblioteki_wykorzystane_w_projekcie}

\subsection{Entity Framework} 
\label{sub:EntityFramework}
\paragraph{}
Entity Framework jest biblioteką służącą do relacyjno-obiektowego mapowania w środowisku ADO.NET. Dzięki zastosowaniu tej biblioteki wraz z narzędziami będącymi integralną częścią języka C\#, można w wygodny sposób zarządzać bazą danych bez konieczności używania języka SQL.

\subsection{AutoMapper} % (fold)
\label{sub:automapper}
\paragraph{} % (fold)
\label{par:}

% paragraph  (end)
AutoMapper jest biblioteką służącą do mapaowania obiektów jednego typu na drugi. Pozwala to w szybki sposób odzielić warstwę aplikacji od warstwy biznesowej na poziomie modleu aplikacji.

\subsection{Unity} % (fold)
\label{sub:unity}
\paragraph{paragraph name} % (fold)
\label{par:paragraph_name}
TODO
% paragraph paragraph_name (end)
% subsection unity (end)

\chapter{Aplikacja internetowa}
\section{Architektura aplikacji} % (fold)
\label{sec:architektura_aplikacji}

\subsection{Wzorzec Model-Widok-Kontroler} % (fold)
\label{sub:wzorzec_model_widok_kontroler}
\paragraph{} % (fold)
\label{par:paragraph_name}
Model-Widok-Kontroler (\textit{ang. Model-View-Controller}) w skrócie MVC, jest wzorcem projektowym rozdzielający aplikację internetową na 3 warstwy : model, widok i kontroler, które komunikują się ze sobą wzajemnie (Rysunek \ref{fig:mvc-pic}). 

\subsubsection{Model}
\paragraph{}
Warstwa modelu odpowiada za reprezentację logiki systemu oraz dostęp do bazy danych. W projekcie za tę część aplikacji odpowiadają dwie biblioteki DLL : \textbf{PI.Service} oraz \textbf{PI.Data}.

\subsubsection{Widok}
\paragraph{}
Widok jest warstwą odpowiedzialną za wyświetlanie interfejsu użytkownika. Najczęściej widoki generowane są na podstawie modelu. W apliakcjach MVC widoki tylko wyświetlają informacje. Ta warstwa znajduje się w bibliotece \textbf{PI.Web}.

\subsubsection{Kontroler}
\paragraph{} 
Kontrolery to komponenty odpowiedzialne za utrzymanie interakcji z użytkownikiem, pracę z modelem i renderowaniem odpowiednich widoków. Kontrolery obsługują rządania użytkowników, konwertują parametry zapytań na modele i przekazują je do kolejnej warstwy. Kontrolery, podobnie jka widoki znajdują się w bibliotece \textbf{PI.Web}.  

\newpage
\begin{figure}[ht]
	\centering
		\includegraphics[width=0.5\linewidth]{assets/03_1.jpg}
	\caption{Schemat graficzny wzorca Model-Widok-Kontroler}
	\label{fig:mvc-pic}
\end{figure}

\subsection{Wzorzec Repozytorium} % (fold)
\label{sub:wzorzec_repozytorium}
\paragraph{} % (fold)
\label{par:}
\textit{Wzorzec Repozytorium} (\textit{Repository Pattern}) to popularna technika mająca na celu podział warstwy biznesowej na dwie części : \textit{Serwis} i \textit{Repozytorium}. \textit{Serwis} jest elementem odpowiedzialnym za logikę aplikacji oraz komunikację między kontrolerem, a repozytorium. \textit{Repozytorium} natomiast jest komponentem, ktorego zadanie polega na komunikacji z bazą danych : zapisywanie, pobieranie, edytowanie i usuwanie danych (model \textit{CRUD}). Takie rozsczepienie poszczególnych elementów pozwal na łatwe testowanie kodu, programowanie zorientowane na testy (\textit{ang. Test Driven Development}), szybką modernizację istniejącej logiki oraz rozbudowę aplikacji.

\paragraph{} % (fold)
\label{par:}
W projekcie \textit{Wzorzec Repozytorium} został zaimplementowany poprzez biblioteki \textbf{PI.Service (Serwis)} i \textbf{PI.Data (Repozytorium)}.
Dane z kontrolerów przekazywane są do \textit{Serwisu} pod postacią \textit{ViewModles} lub typów prymitywnych. \textbf{PI.Service} odpowiada za odpowiednią logikę przetwarzania danych, konwersji \textit{ViewModels} na \textit{Models} (przy wykorzystaniu narzędzia \textit{AutoMapper \footnote{AutoMapper - Sekcja \ref{sub:automapper}}}) i przekazaniu ich do \textit{Repozytorium}.

\paragraph{} % (fold)
\label{par:}
Funkcją biblioteki \textbf{PI.Data} jest odbieranie danych z \textit{Serwisu} i dokonanie odpowiednich operacji bazodanowych przy użyciu \textit{EntityFramework \footnote{EntityFramework - Sekcja \ref{sub:EntityFramework}}}

\paragraph{} % (fold)
\label{par:}
Schemat działania logiki biznesowej w oparciu o \textit{Wzorzec Repozytorium} został przedstawiony na Rysunku \ref{fig:repository-pattern} .

\begin{figure}[ht]
	\centering
		\includegraphics[width=0.8\linewidth]{assets/repository_pattern.png}
	\caption{Schemat przepływu danych przy wykorzystaniu Wzorca Repozytorium}
	\label{fig:repository-pattern}
\end{figure}

\subsection{Odwrócenie sterowania} % (fold)
\label{sub:odwr_cenie_sterowania}
\paragraph{} 
\textit{Odwrócenie sterowania} (\textit{ang. Inversion of Control}) to pradygmat odpowiedzialny za przeniesienie na zaewnątrz obiektu odpowiedzialnego za kontrolę niektórych czynności. Termin ten jest często utożsamiany z \textit{Wstrzykiwaniem zależności (ang. Dependency Injection)}, jednak jest to tylko jedna z realizacji \textit{Odwrócenia sterowania}. 

\paragraph{} % (fold)
\label{par:}
\textit{Wstrzykiwanie zależności} zosatało zrealizowane w projekcie za pomocą biblioteki \textit{Unity \footnote{Unity - Sekcja \ref{sub:unity}}} na dwóch poziomach :
\begin{itemize}
	\item W warstwie Modelu - wszytskie serwizy i repozytoria, oraz kontekst bazodanowy wstrzykiwane są jako parametry w konstruktorze
	\item W warstwie Kontrolorów - wstrzykiwany jest dostawca serwisów
\end{itemize} 

\paragraph{} % (fold)
\label{par:}
Takie rozdzielenie wstrzykiwania serwisów pozwoliło na całkowite odseparowanie logiki bazodanowej i elementów za nią odpowiedzialnych od warstwy aplikacji.

\subsection{Bezpieczeństwo} % (fold)
\subsubsection{Autoryzacja}
\paragraph{} % (fold)
\label{par:paragraph_name}

% paragraph paragraph_name (end)
Do autoryzacji użytkowników, w projekcie wykorzystany zsotał popularny z technologii ASP.NET system \textit{Forms authentication}. \textit{Forms authentication} wykorzystuje znacznik uwierzytelniający, który zostaje utworzony w momencie gdy użytkownik zaloguje się na stronie. Zancznik \textit{Forms authentication} zwykle przechowywane jest wewnątrz szyfrowanego ciasteczka.

\paragraph{} % (fold)
\label{par:}
W momencie gdy użytkownik będzie chciał uzyskać dostęp do strony wymagającej uwierzytelnienia, a nie przechodził wcześniej procesu logowania, zostanie przekierowany na zdefiniowaną w pliku konfiguracyjnym stronę logowania, na której będzie mógł wprowadzić nazwę użytkownika i hasło. Te dane są następnie przekazane do serwera, który sprawdza czy takowy użytkownik istnieje w bazie danych. Po pomyślnej weryfikacji danych, użytkownik zostaje uwierzytelniony, stworzony zostaje znacznik i następuje przekierowanie na stronę główną aplikacji.
% paragraph  (end)

\subsubsection{Przechowywanie hasła}
\paragraph{} % (fold)
\label{par:}
Przechowywane w systemie hasła, szyfrowane są poprzez wygenerowanie funkcji skrótu \textit{SHA-2} o wilekości 256 sklejeonego z \textit{Solą} hasła. \textit{Sól} stanowi wygenerowany w momencie rejestracji \textit{GUID - identyfikator globalnie unikatowy}. W aplikacji został wykorzystany algorytm \textit{SHA-2}, ponieważ jest zancznie silniejszy od swoich poprzedników z rodziny \textit{SHA-1}, w którym zidentyfikowano luki w bezpieczeństwie. Ogólny algorytm szyfrowania wygląda następująco :

\paragraph{} % (fold)
\label{par:}

% paragraph  (end)
\lstset{language=Java,caption={Szyfrowanie hasła},label=DescriptiveLabel}

\begin{lstlisting}
		SHA256(concat(password,salt))
\end{lstlisting}

\paragraph{} % (fold)
\label{par:}
Dzięki zastosowaniu \textit{soli} zmniejszone zostaje prawdopodobieństwo złamania hasła popularnymi atakami : brutal force, atakiem słownikowym czy wykorzystującym tablice tęczowe.
% paragraph  (end)

% paragraph  (end)
% subsection autoryzacja (end)
% section section_name (end)

\section{Baza Danych} % (fold)
\label{sec:baza_danych}
\paragraph{} % (fold)
\label{par:}
Baza danych, do której dostęp ma aplikacja internetowa działa na serwerze \textit{Microsoft SQL Server 2012 \footnote{MS SQL 2012 - Sekcja \ref{sub:mssql}}}. Do zarządzania danymi wykorzystany został wykorzystana bibliotek \textit{EntityFramework \footnote{EntityFramework - Sekcja \ref{sub:EntityFramework}}}. W bazie danych przechowywane są informacje o profilu użytkownika, ustawieniach konta oraz zarejestrowane przez aplikację mobilną współrzędne GPS.

\paragraph{} % (fold)
\label{par:}
W tworzeniu i zarządzaniu bazą danych wykorzystane zostało podejście \textit{Code First}, polegające na modelowaniu bazy danych pod postacią klas języka \textit{C\#}, a następnie wygenerowaniu gotowej bazy danych przy pomocy EntityFramework. By zabezpieczyć dane składowane w bazie danych przed usunięciem ich w wyniku zmiany modelu w kodzie projektu, użyty został mechanizm migracji 
Schemat bazy danych przedstawiony został na Rysunku \ref{fig:database} .


\label{sub:schemat_bazy_danych}
\begin{figure}[ht]
	\centering
		\includegraphics[width=1\linewidth]{assets/database.png}
	\caption{Schemat bazy danych}
	\label{fig:database}
\end{figure}

% subsection schemat_bazy_danych (end)
% paragraph  (end)

% section baza_danych (end)

\section{Warstwa prezentacji} % (fold)
\label{sec:warstwa_aplikacji}
\paragraph{} % (fold)
\label{par:}
By aplikacja internetowa była przyjazna i łatwa w użyciu, należy zadbać o przejrzysty, łatwy w obsłudze i estetyczny interfejs. W tym celu wykorzystana została popularna biblioteka stylów CSS Twitter-Bootstrap \footnote{http://twitter.github.com/bootstrap/}. Dzięki dużej ilośći gotowych komponentów, można w łatwy i szybki sposób projektować wygląd strony internetowej, spełniający oczekiwania współczesnego użytkownika. Twitter-Bootstrap to nie tylko biblioteka styli CSS, ale również zbiór elementów JavaScript takich jak okna modalne czy zakładki.

\paragraph{} % (fold)
\label{par:}
W celu poprawienia funkcjonalności aplikacji i interakcji z użytkownikiem wykorzystane zotały dodatkowe moduły takie jak :
\begin{itemize}
\item jQuery Data Table \footnote{http://datatables.net/}
\item pNotify \footnote{http://pinesframework.org/pnotify/}
\item jQuery DatePicker \footnote{http://jqueryui.com/}
\item jQuery DateTimePicker \footnote{http://trentrichardson.com/examples/timepicker/}
\item FullCalendar \footnote{http://arshaw.com/fullcalendar/}
\item Google Maps JavaScript API v3 \footnote{https://developers.google.com/maps/documentation/javascript/}
\end{itemize}


\section{Funkcjonalności} % (fold)
\label{sec:funkcjonalno_ci}

\subsection{Treningi} % (fold)
\label{sub:treningi}

\paragraph{}
Moduł do obsługi treningów jest najważniejszym elementem projektu. Pozwala on na śledzenie dodawanie i archiwizowanie danych dotyczących. Zapisane w systemie treningi wyświetlane są na kalendarzu (Rysunek \ref{fig:workout_main}), przez co można w uporządkowany sposób zarządzać treningami, planować następne sesje, archiwizować dane i wyszukiwać wcześniejsze rekordy. Po kliknięciu na wybrany przez nas trening, wyświetlone zostaje modalne okno zawierjące wsyztskie inforamcje dotyczące aktywności : typ, data początku i końca, czas, dystans i jeżeli w bazie danych są przechowane współrzędne GPS, to również mapa z zaznaczoną trasą (Rysunek \ref{fig:workout_map}). Dodawanie treningu może odbywać się na dwa sposoby : poprzez synchronizację z danymi przechowywanymi przez aplikację mobilną lub dodając rekord bezpośrednio poprzez interfejs aplikacji internetowej (Rysunek \ref{fig:add_workout}).


\begin{figure}[ht]
	\centering
		\includegraphics[width=1\linewidth]{assets/workouts_main.png}
	\caption{Widok strony agregującej dodane rekordy}
	\label{fig:workout_main}
\end{figure}

\begin{figure}[ht]
	\centering
		\includegraphics[width=1\linewidth]{assets/workout_map.png}
	\caption{Widok treningu z zaznaczoną trasą}
	\label{fig:workout_map}
\end{figure}

\begin{figure}[ht]
	\centering
		\includegraphics[width=1\linewidth]{assets/add_workout2.png}
	\caption{Dodawanie treningu poprzez aplikację internetową}
	\label{fig:add_workout}
\end{figure}

\section{Funkcje portalu społecznościowego} % (fold)
\label{sec:profil_u_ytkownika}
\paragraph{} % (fold)
\label{par:}
By zapewnić powodzenie aplikacji w dzisiejszych czasach należy zadbać również o dodatkowe funkcjonalności aplikacji, pozwalające na interakcję z użytkowników między sobą. Możliwość podąrzania za użytkownikami, przeglądanie ich aktywności, osiągnięć, możliwość komunikacji poprzez wysyłanie wiadomości sprawia, iż aplikacja staje się małym portalem społecznościowym. 
\paragraph{} % (fold)
\label{par:}

% paragraph  (end)
Jednak by zadowolić użytkowników, chcących wykorzystywać serwis tylko i wyłącznie jako internetowy dziennik treningów, zaimplementowane zostały różne poziomy udostępniania treści innyym użytkownikom portalu. Podstawowe wiadomości jak imię i nazwisko oraz nazwa użytkownika są widoczne dla wszytskich, jednak resztę informacji w tym adres email można ukryć, wyłączając w widoku edycji profilu flagę "Publiczny profil". Również istnieje możliwość udostępniania swoich treningów innym użytkownikom na trzech poziomach : dostępne dla wszytskich, dostępne dla użytkowników podążających, niewidoczne dla nikogo.
% paragraph  (end)

\begin{figure}[ht]
	\centering
		\includegraphics[width=1\linewidth]{assets/profil.png}
	\caption{Profil użytkownika}
	\label{fig:profil}
\end{figure}


\begin{figure}[ht]
	\centering
		\includegraphics[width=1\linewidth]{assets/edit.png}
	\caption{Widok edycji profilu}
	\label{fig:add_workout}
\end{figure}

\begin{figure}[ht]
	\centering
		\includegraphics[width=1\linewidth]{assets/chat.png}
	\caption{Widok wysyłania wiadomości z historią w formie czatu}
	\label{fig:add_workout}
\end{figure}



\bibliographystyle{plalpha}

\end{document}
