\chapter{Wprowadzenie}

\section{Wstęp} % (fold)
\label{sec:wst_p}
Popularyzacja sportu wśród społeczeństwa jest jednym z głównych celów dzisiejszych kampanii na rzecz zdrowego stylu życia. 
% section wst_p (end)

\section{Cel pracy}
\label{sec:cel_pracy}
\paragraph{}
Celem niniejszego projektu inżynierskiego jest zaprojektowanie oraz zaimplementowanie systemu do wspomagania treningów areobowych. Główną częścią systemu jest aplikacja internetowa wraz z aplikację mobilną. By zapewnić możliwość przesyłania danych do serwera, konieczne jest stworzenie API, będącego pośrednikiem między aplikacją mobilną, a bazą danych.
System ma pozwolić użytkownikom zarządzać treningami, prezentować swoje osiągnięcia innym osobom korzystającym z portalu i swobodnie komunikować się z nimi.
% paragraph  (end)
\paragraph{} % (fold)
\label{par:}
Podstawowe wymagania, które ma spełniać projekt:
\begin{itemize}
	\item tworzenie i edycja treningów,
	\item tworzenie i edycja profili użytkowników,
	\item komunikacja pomiędzy użytkownikami,
	\item synchronizacja danych z aplikacji mobilnej,
	\item zarządzanie prywatnością profilu użytkownika,
	\item możliwość przeglądania treningów innych użytkowników,
	\item stworzenie formy małego portalu społecznościowego.
\end{itemize}

\paragraph{} % (fold)
\label{par:}
Zakres prac wykoannych w projekcie:
\begin{itemize}
  \item projekt systemu - rozdział \ref{cha:projektsys}
	\item implementacja aplikacji internetowej - rozdział \ref{cha:webapp}
	\item implementacja aplikacji mobilnej - rozdział \ref{cha:mobileapp}
	\item implementacja REST Api - rozdział \ref{cha:restapi}
\end{itemize}
% paragraph  (end)
% paragraph  (end)

\section{Istniejące rozwiązania}
\paragraph{}
Na rynku dostępnych jest wiele rozwiązań pozwalających na zarządzanie treningami np. \textit{Endomondo} lub \textit{Nike Running}. Jednak większość z nich jest aplikacjami komercyjnymi, których darmowe produkty obarczone są dużą ilością reklam lub ograniczoną liczbą dostępnych funkcjonalności. Ponadto aplikacje te często zatracają swoją idę wspierania treningów i stają się kolejnymi portalami typu Social Media. 
\paragraph{} % (fold)
\label{par:}

% paragraph  (end)
Celem tego projektu jest stworzenie systemu, oferującego podobne możliwości jak oferowane na rynku rozwiązania, nie tracąc przy tym na prostocie użytkowania i przejrzystości interfejsu użytkownika.
