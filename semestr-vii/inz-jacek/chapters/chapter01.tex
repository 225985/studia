\chapter{Wprowadzenie}

\section{Cel pracy}
\label{sec:cel_pracy}
\paragraph{}
Celem niniejszego projektu inżynierskiego jest zaprojektowanie oraz zaimplementowanie systemu do wspomagania treningów areobowych. Główną częścią systemu jest aplikacja internetowa wspomagana przez aplikację mobilną. By zapewnić możliwość przesyłania danych do serwera, konieczne jest stworzenie API, będącego pośrednikiem między aplikacją mobilną, a bazą danych.
System ma pozwolić użytkownikom zarządzać treningami, prezentować swoje osiągnięcia innym osobom korzystjącym z portalu i swobodnie komunikować się z nimi.
% paragraph  (end)
\paragraph{} % (fold)
\label{par:}
Podstawowe wymagania, które ma spełniać projekt :
\begin{itemize}
	\item Tworzenie i edycja treningów
	\item Tworzenie i edycja profili użytkowników
	\item Komunikacja pomiędzy użytkownikami
	\item Synchronizacja danych z aplikacji mobilnej
	\item Zarządzanie prywatnością profilu użytkownika
	\item Możliwość przeglądania treningów innych użytkowników
	\item Stworzenie formy małego portalu społecznościowego
\end{itemize}
% paragraph  (end)

\section{Istniejące rozwiązania}
\paragraph{}
Na rynku dostępne jest wiele rozwiązań pozwalających na zarządzanie treningami. Jednak większość z nich jest aplikacjami komercyjnymi, których darmowe produkty obarczone są dużą liczbą reklam lub ograniczoną liczbą dostępnych funckcjonalności. Ponadto aplikacje te często zatraciły swoją idę wspierania treningów i stały się kolejnymi portalami typu Social Media. Celem tego projektu jest stworzenie systemu, oferującego podobne możliwośći jak oferowane na rynku rozwiązania, nie tracąc przy tym na prostocie użytkowania i przejrzystościu interfejsu użytkownika.
