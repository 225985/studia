\chapter{Aplikacja mobilna} % (fold)
\label{cha:aplikacja_mobilna}

\paragraph{} % (fold)
\label{par:}
Aplikacja mobilna jest integralną częścią całego systemu do wspierania treningów areobowych. Jej zadaniem jest pobieranie informacji o przebytej trasie za pomocą odbiornika GPS, obliczanie dystansu i czasu trwania treningu, przetrzymywaniu informacji w bazie danych oraz synchronizacja rekordów z apikacją internetową. 
% paragraph  (end)

\section{Architektura system} % (fold)
\label{sec:architektura_system}

% section architektura_system (end)
\paragraph{} % (fold)
\label{par:}
Aplikacja została napisana w języku Java i działa na mobilnej platformie Android w wersji 4.0.3. Najważniejszą częścia programów napisanych na system Android jest \textit{Aktywność (ang. Activity)}, charakteryzująca się określonym cyklem życia. Sposób w jaki zarządza się cyklem życia aplikacji, uruchamia odpowiednie zdarzenia w określonych momentach i zachowanie aktywności ma fundamentalny wpływ na działanie całej aplikacji.

\subsection{Aktywności} % (fold)
\label{sub:aktywno_ci}

% subsection aktywno_ci (end)
\paragraph{} % (fold)
\label{par:}
Aktywności wyróżniają się czterema stanami :
\begin{itemize}
	\item Aktywna (ang. active) lub chodząca (ang. running) - gdy aktywność znajduje się w warstwie szczytowej aplikacji
	\item Wstrzymana (ang. paused) - gdy aktywność jest widoczna, ale nie znajduje się na pierwszym planie. W tym stanie może zostać zakończona przez system, w przypadku barku pamięci
	\item Zatrzymana (ang. stopped) - gdy aktywność zostanie zasłonięta przez inna aktywność. W tym stanie system przechowuje pełną informację na temat aktywności, lecz nie jest ona widoczna dla użytkownika
	\item Gdy aplikacja jest w stanie wstrzymania lub zatrzymania, może zostać usunięta z pamięci systemu. W momencie ponownego uruchomienia aktywności, będzie musiała być stworzona od początku.
\end{itemize}

\paragraph{} 
Pełny cykl życia aktywności został przedstawiony na Rysunku \ref{fig:activity_lifecycle}

\begin{figure}[ht]
	\centering
		\includegraphics[width=0.7\linewidth]{assets/activity_lifecycle.png}
		\caption{Pełny cykl życia aktywności}
	\label{fig:activity_lifecycle}
\end{figure}

\subsection{Moduł GPS} % (fold)
\label{sub:modu_gps}

% subsection modu_gps (end)
\paragraph{} % (fold)
\label{par:}
Obsługa modułu GPS polega na zaimplementowaniu interfejsów odpowiedzialnych za szczytywanie informacji (\textit{LocationListener}) na temat położenia i badanie stanu odbiornika GPS (\textit{GpsStatusListener}). Oba interfejsy są uruchamiane w zdarzeniu \textit{onCreate()} głównej aktywności. 

\subsection{Baza danych} % (fold)
\label{sub:baza_danych}
Informacje na temat treningów przechowywane są w bazie danych SQLite \footnote{SQLLite - Sekcja \ref{sub:sqlite}}. Model bazy danych (Rysunek \ref{fig:sqlite-model}) stanowią dwie tabele : \textit{Workout} i \textit{WorkoutDetails}. Pierwsza z nich służy do przechowywania podstawowych informacji o treningu : początek, koniec, czas trwania, dystans. W drugiej tabeli natomiast przechowywane są współrzędne geograficzne szczytane z GPS.
% subsection baza_danych (end)

\subsection{Autoryzacja} % (fold)
\label{sub:}
Po wprowadzeniu nazwy użytkownika i hasła w formularzu na aktywności Login, wysłane zostaje rządanie autoryzacji do REST API. Po pozytywnej weryfikacji danych zwrócona zostaje odpowiedź o statusie HTTP 200 wraz z tokenem, służącym do dalszej autoryzacji aplikacji. Zweryfikowane dane użytkownika (login i hasło) przechowywane są w \textit{SharedPreferences}. Szczegółowy opis autoryzacji znajduje się w sekcji \ref{sub:autoryzacja}.

\subsection{Wyświetlanie mapy} % (fold)
\label{sub:wy_wietlanie_mapy}
\paragraph{} % (fold)


W celu wyświetlenia przebytej trasy w aplikacji mobilnej wykorzystane zostało \textit{Google Map Android API}. Pozwala ono na pobranie z serwisu Google mapy o wskazanej pozycji i przyblizeniu, a także na rysowaniu lini pomiędzy punktami na jednej z jej warstw. 

\paragraph{}
Aby wyznaczyć środek wyświetlanego obszaru, trzeba znaleźć cztery wartości : maksymalną i minimalną długość i szerokość. Punkt środka mapy wyznacza się z następującego wzoru :

\begin{equation}
	CenterLatitude = \frac{MaxLatitude + MinLatitude}{2}
\end{equation}

\begin{equation}
		CenterLongitude = \frac{MaxLongitiude + MinLongitiude}{2}
\end{equation}

\paragraph{} % (fold)
\label{par:}
NAstępnym etapem jest pobranie z bazy danych współrzędnych zarejestrowanych punktów i połączenie ich za pomocą linii.
% paragraph  (end)