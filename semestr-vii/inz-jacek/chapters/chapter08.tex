\chapter{Podsumowanie} % (fold)
\label{cha:podsumowanie}

\section{Realizacja założeń} % (fold)
\label{sec:realizacja_za_o_e_}
\paragraph{} % (fold)
\label{par:}

% paragraph  (end)
Przedstawiony projekt w pełni realizuje załóżenia przedstawione w Sekcji \ref{sec:cel_pracy}. Jest w pełni funkcjonalnym i działającym systemem, pozwalającym na kontrolowanie swoich treningów oraz interakcje z innymi użytkownikami portalu.
% section realizacja_za_o_e_ (end)

\section{Napotkane problemy} % (fold)
\label{sec:napotkane_problemy}
\paragraph{} % (fold)
\label{par:}
Głównym problemem napotkanym podczas implementacji i testowania systemu była konfiguracja serwera \textit{IIS}, która na domyślnych parametrach nie zezwalała na sposoby autoryzacji zastosowane w projkecie.
% paragraph  (end)

% section napotkane_problemy (end)

\section{Kierunki rozwoju} % (fold)
\label{sec:kierunki_rozwoju}

\paragraph{} % (fold)
\label{par:}
By zapewnić sukces marketingowy projektowi, należałoby zaimplementować aplikacje mobilne na inne systemy operacyjne niż tylko \textit{Andorid}. W dzisiejszych czasach najpopularniejszy,i systemami operacyjnymi są : iOS, Android, BlackBerry, w ostatnich czasach również Windows 7.5 i 8. Nie można również zapomnieć o użytkownikach posiadających starsze telefony działające na systemie Symbian OS.

\paragraph{} % (fold)
W celu zapewnienia bezpieczeństwa przesyłanych danych pomiędzy aplikacją mobilną, a REST API, wykupiony powinien zostać certyfikat SSL, pozwalający na szyfrowanie przesyłanych danych.

\paragraph{} % (fold)
 \label{par:}
 Ważnym elementem jest również niezawodność działania aplikacji internetowej. By zapewnić skalowalność serwera i łatwą konfigurowalność parametrów instancji na jakich działa system, wdrożony powinien zostać na platformę Windows Azure, do czego został dostosowany.