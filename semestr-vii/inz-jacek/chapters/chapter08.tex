\chapter{Podsumowanie} % (fold)
\label{cha:podsumowanie}

\section{Realizacja założeń} % (fold)
\label{sec:realizacja_za_o_e_}
\paragraph{} % (fold)
\label{par:}

% paragraph  (end)
Przedstawiony projekt w pełni realizuje założenia przedstawione w Sekcji \ref{sec:cel_pracy}. Jest całkowicie funkcjonalnym i działającym systemem, pozwalającym na kontrolowanie swoich treningów oraz interakcje z innymi użytkownikami portalu.
% section realizacja_za_o_e_ (end)

\section{Napotkane problemy} % (fold)
\label{sec:napotkane_problemy}
\paragraph{} % (fold)
\label{par:}
Głównym problemem napotkanym podczas implementacji i testowania systemu była konfiguracja serwera \textit{IIS}. Przy użyciu domyślnych parametrów serwer nie zezwalał na sposoby autoryzacji zastosowane w projekcie.
% paragraph  (end)

% section napotkane_problemy (end)

\section{Kierunki rozwoju} % (fold)
\label{sec:kierunki_rozwoju}

\paragraph{} % (fold)
\label{par:}
By zapewnić sukces marketingowy projektu, należałoby zaimplementować aplikacje mobilne na inne systemy operacyjne niż tylko \textit{Andorid}. W dzisiejszych czasach jednymi z najpopularniejszych systemów operacyjnych są: iOS, Android, BlackBerry OS, a w ostatnich czasach również Windows Phone 7.5 i 8. Nie można także zapomnieć o użytkownikach posiadających starsze telefony działające na systemie Symbian OS, Bada OS.

\paragraph{} % (fold)
W celu zapewnienia bezpieczeństwa przesyłanych danych pomiędzy aplikacją mobilną, a REST API, system powinien zostać uzupełniony o możliwość szyfrowania danych poprzez wykorzystanie certyfiaktów \textit{SSL}.

\paragraph{} % (fold)
 \label{par:}
 Ważnym elementem jest również niezawodność działania aplikacji internetowej. By zapewnić skalowalność serwera i łatwą konfigurowalność parametrów instancji na jakich działa system, powinien zostać wdrożony na platformę Windows Azure, do czego został dostosowany.

 \paragraph{} % (fold)
 \label{par:}
 
 Z uwagi na niską jakość montowanych w urządzenaich mobilnych odbiorników GPS, a w konsekwencji niską dokładność odczytywanych pozycji, należałoby w celu wyświetlania na mapie przebytej trasy zaimplementować filtr Kalmana. Pozwoliłoby to na pokazanie wygładzonej trasy z mniejszą ilością niedokładności spowodowanych dużą rozdzielczością odbiornika.