\documentclass[wide,a4paper,titlepage,12pt]{article}
\usepackage{polski}
\usepackage[utf8]{inputenc}
\usepackage{setspace}
\usepackage{graphicx}
\usepackage{xcolor,listings}

\title{Zastosowanie systemów wbudowanych}
\author{Tymon Tobolski {\small(181037)}\\ Jacek Wieczorek {\small(181043)}}

\makeatletter
\renewcommand{\maketitle}{
  \begin{titlepage}
    \begin{center}
      \vspace*{3cm}
      \LARGE \@title \par
      \vspace{2cm}
      \textit{\small Autor:}\par
      \normalsize \@author\par \normalsize
      \vspace{3cm}
      \textit{\small Prowadzący:}\par
      dr inż. Jacek Majewski \par
      \vspace{2cm}
      Wydział Elektroniki\\ IV rok\\ Pn 08.00 - 11.00\par

    \end{center}
  \end{titlepage}
}
\makeatother
\setlength{\parindent}{0em}
\onehalfspacing


\begin{document}
\maketitle

\section{Cel projektu} % (fold)
\label{sec:cel_projektu}
\paragraph{} % (fold)
\label{par:}

% paragraph  (end)
Celem projektu było stworzenie systemu do obsługi balona meteorologicznego w oparciu o platformę \textit{Raspberry PI}.
% section cel_projektu (end)
\section{Wykorzystany sprzęt} % (fold)
\label{sec:potrzebyn_sprz_t}
\begin{itemize}
  \item Komputer Raspberry PI w wersji z 256 MB pamięci RAM
  \item Karta pamięci SD o minimalnej wielkości 4GB
  \item Odbiornik Bluetooth na USB
  \item Odbiornik GPS komunikujący się po Bluetooth 
\end{itemize}
% section potrzebyn_sprz_t (end)

\section{Instalacja oprogramowania i konfiguracja} % (fold)
\label{sec:instalacja_oprogramowania_i_konfiguracja}
\paragraph{} % (fold)
\label{par:}
W celu uruchomienia projketu należy w systemi Raspberrian zainstalować potrzebne biblioteki następującymi komendami: 
\lstset{language=bash,caption={Instalacja bibliotek},label=DescriptiveLabel}

\begin{lstlisting}
sudo apt-get update
sudo apt-get upgrade
sudo apt-get install bluetooth bluez-utils blueman ruby1.9.1 ffmpeg
\end{lstlisting}

\paragraph{} % (fold)
\label{par:}
By uruchomić projekt należy wykonać komendę:
\lstset{language=bash,caption={Uruchomienie projektu},label=DescriptiveLabel}

\begin{lstlisting}
                        ruby app.rb
\end{lstlisting}

\lstset{language=bash,caption={Przydatne komendy},label=DescriptiveLabel}

\begin{lstlisting}
service bluetooth status - sprawdzenie stanu odbiornika bluetooth
hcitool scan - wyszukanie nadajnikow bluetooth
\end{lstlisting}
\end{document}

