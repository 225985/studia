\documentclass[wide,a4paper,titlepage,12pt] {article}
\usepackage{polski}
\usepackage[UTF8]{inputenc}
\usepackage{listings}
\usepackage{slashbox}
\usepackage[table]{xcolor}
\usepackage{graphicx,pdflscape}
\usepackage{placeins}


\title{Grafika komputerowa}
\author{Jacek Wieczorek (181043)}


% Title page layout (fold)
\makeatletter
\renewcommand{\maketitle}{
\begin{titlepage}
  \begin{center}
    \vspace*{3cm}
    \LARGE \@title \par
    \vspace{2cm}
    \textit{\small Autor:}\par
    \normalsize \@author\par \normalsize
    \vspace{3cm}
    \textit{\small Prowadzący:}\par
    Dr inż. Tomasz Kapłon \par
    \vspace{2cm}
    Wydział Elektroniki\\ III rok\\ Pn TP 08.15 - 11.00\par
    \vspace{4cm}
    \small \@date
  \end{center}
\end{titlepage}
}
\makeatother
% Title page layout (end)

% \putcode{source}{language}
\newcommand{\putcode}[2] {
 \lstset{ 
    language=#2,                % choose the language of the code
    basicstyle=\scriptsize,       % the size of the fonts that are used for the code
    numbers=left,                   % where to put the line-numbers
    numberstyle=\scriptsize,      % the size of the fonts that are used for the line-numbers
    stepnumber=10,                   % the step between two line-numbers. If it's 1 each line 
    numbersep=9pt,                  % how far the line-numbers are from the code
    showspaces=false,               % show spaces adding particular underscores
    showstringspaces=false,         % underline spaces within strings
    showtabs=false,                 % show tabs within strings adding particular underscores
    }
    \lstinputlisting{#1}
}
% \putpicture{source}{caption}{size}{label}
\newcommand{\putpicture}[4] {
    \begin{figure}[htbp]
        \begin{center}
            \includegraphics[width=#3]{#1}
            \caption{#2}
            \label{#4}
        \end{center}
    \end{figure}
}


\begin{document}
\maketitle

\section{Cel laboratorium}
\paragraph{}
Celem ćwiczenia było poznanie podstawowych technik teksturowania powierzchni z wykorzystaniem \textit{OpenGL} i \textit{GLUT}.

\section{Współne funkcje programu dla każdego z zadań}
\subsection{Wczytanie tekstury}
\putcode{code/dupa.cpp}{c++}
\subsection{Sterowanie za pomocą myszki}

\section{Zadanie 1}
\paragraph{} 
Pierwsze zadanie polegało na narysowaniu trójkąta i wypełnbienu go dowolnie wybraną teksturą zapisaną w pliku o \textit{TGA}.
\subsection{Przykładowy obraz}
\putpicture{img/img1.PNG}{Tekstura nałożona na trójkąt}{\textwidth}{}

\section{Zadanie 2}
\paragraph{} % (fold)
Celem drugie zadania było zbudowanie i teksturowanie jednostronne ostrosłupa o kwadratowej podstawie.
\subsection{Przykładowe obrazy}
\putpicture{img/img2_0.PNG}{Tekstura nałożona na piramidę}{\textwidth}{}
\putpicture{img/img2_1.PNG}{Tekstura nałożona na podstawę piramidy}{\textwidth}{}
\putpicture{img/img2_2.PNG}{Tekstura nałożona na jeden z boków piramidy}{\textwidth}{}


\section{Zadanie 3}
\paragraph{} 
Ostatnie zadanie polegało na nałożeniu na jajko wykorzystywane podczas poprzednich laboratorów tekstur.
\subsection{Przykładowy obraz}
\putpicture{img/img3_0.PNG}{Tekstura nałożona na jajko}{\textwidth}{}
\putpicture{img/img3_1.PNG}{Tekstura nałożona na jajko}{\textwidth}{}
\putpicture{img/img3_2.PNG}{Tekstura nałożona na jajko}{\textwidth}{}
\putpicture{img/img3_2.PNG}{Tekstura nałożona na pomniejszone jajko}{\textwidth}{}

\section{Wnioski}
\paragraph{}



\end{document}

