\documentclass[wide,a4paper,titlepage,12pt] {article}
\usepackage{polski}
\usepackage[UTF8]{inputenc}
\usepackage{listings}
\usepackage{slashbox}
\usepackage[table]{xcolor}
\usepackage{graphicx,pdflscape}
\usepackage{placeins}
\usepackage{reportshelper}




\begin{document}
\maketitle{Grafika Komputerowa}{Jacek Wieczorek}{Dr inż. Tomasz Kapłon}{Wydział Elektroniki}{III rok}{Pn TP 8.15 - 11.00}

\section{Cel laboratorium}
\paragraph{}
Celem ćwiczenia było poznanie podstawowych technik teksturowania powierzchni z wykorzystaniem \textit{OpenGL} i \textit{GLUT}.

\section{Współne funkcje programu dla każdego z zadań}
\subsection{Wczytanie tekstury}
\putcode{code/p1.cpp}{c++}
\subsection{Sterowanie za pomocą myszki}
\putcode{code/p2.cpp}{c++}
\subsection{Funkcja \textit{MyInit}}
\putcode{code/p3.cpp}{c++}

\newpage
\section{Zadanie 1}
\paragraph{} 
Pierwsze zadanie polegało na narysowaniu trójkąta i wypełnbienu go dowolnie wybraną teksturą zapisaną w pliku o \textit{TGA}.
\subsection{Kod programu}
\putcode{code/trojkat.cpp}{c++}
\subsection{Przykładowy obraz}
\putpicturescale{img/img1.PNG}{Tekstura nałożona na trójkąt}{0.8}{}


\newpage
\section{Zadanie 2}
\paragraph{} % (fold)
Celem drugie zadania było zbudowanie i teksturowanie jednostronne ostrosłupa o kwadratowej podstawie.
\paragraph{}
By umozliwić sprawdzenie jednostronnego teksturowania piramidy, zaimplementowana została mozliwość wybierania za pomoca klawiatury, która ściana piramidy ma być wyświetlana.
\subsection{Kod programu}
\subsubsection{Sterowanie klawiaturą}
\putcode{code/piramida_1.cpp}{c++}
\subsubsection{Rysowanie piramidy}
\putcode{code/piramida_2.cpp}{c++}
\newpage
\subsection{Przykładowe obrazy}
\putpicturescale{img/img2_0.PNG}{Tekstura nałożona na piramidę}{0.45}{}
\putpicturescale{img/img2_1.PNG}{Tekstura nałożona na podstawę piramidy}{0.45}{}
\putpicturescale{img/img2_2.PNG}{Tekstura nałożona na jeden z boków piramidy}{0.45}{}


\newpage
\section{Zadanie 3}
\paragraph{} 
Ostatnie zadanie polegało na nałożeniu na jajko wykorzystywane podczas poprzednich laboratoriów tekstur.
\subsection{Kod programu}
\putcode{code/jajko_1.cpp}{c++}
\newpage
\subsection{Przykładowy obraz}
\putpicturescale{img/img3_0.PNG}{Tekstura nałożona na jajko}{0.45}{}
\putpicturescale{img/img3_1.PNG}{Tekstura nałożona na jajko}{0.45}{}


\newpage
\section{Wnioski}
\paragraph{}
Tesksturowanie obiektów z wykorzystaniem biblioteki \textit{OpenGL} z rozszerzeniem \textit{GLUT} nie jest trudnym zadaniem. Po zapoznaniu sie z listą parametrów i możliwości jakie daje nam wyżej wymieneiona biblioteka, możemy w praktycznie dowolny sposób nakładać tekstury na obiekty, definiować czy mają być jednostronne czy dwustronne, ładowac dowolne obrazy.



\end{document}

