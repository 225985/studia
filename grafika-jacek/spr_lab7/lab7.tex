\documentclass[wide,a4paper,titlepage,12pt] {article}
\usepackage{polski}
\usepackage[UTF8]{inputenc}
\usepackage{listings}
\usepackage{slashbox}
\usepackage[table]{xcolor}
\usepackage{graphicx,pdflscape}
\usepackage{placeins}
\usepackage{reportshelper}




\begin{document}
\maketitle{Grafika Komputerowa\\ Ray Tracing}{Jacek Wieczorek}{Dr inż. Tomasz Kapłon}{Wydział Elektroniki}{III rok}{Pn TP 8.15 - 11.00}

\section{Cel projektu}
\paragraph{}
Celem projektu było zaimplementowanie rekursywnej algorytmu śledzenia promieni \textit{Ray Tracing}. 

\section{Ray Tracing}
\paragraph{}
\textit{Ray Tracing} opiera się ona na analizowaniu poszczególnych promieni emitowanych przez źródło światła w kierunku od źródła światła do rzutni. Następnie wylicza się kolejne kierunki odbicia analizowanego promienia od ścian obiektów, aż do wyznaczania kierunku ostatniego odbicia promienia. Prosta wyznaczana przez ostatni kierunek odbicia analizowanego promienai przecina rzutnię, bądź nie.

\section{Opis algorytmu}
\paragraph{}
 




\end{document}

