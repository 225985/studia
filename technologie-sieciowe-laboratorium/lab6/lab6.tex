\documentclass[wide,a4paper,titlepage,12pt] {article}
\usepackage{polski}
\usepackage{float}
\usepackage[utf8]{inputenc}
\usepackage{listings}
\usepackage{slashbox}
\usepackage[table]{xcolor}
\usepackage{graphicx,pdflscape}
\usepackage{placeins}


\title{Technologie sieciowe 2}
\author{Tymon Tobolski (181037)\\ Jacek Wieczorek (181043)}

% Title page layout (fold)
\makeatletter
\renewcommand{\maketitle}{
\begin{titlepage}
  \begin{center}
    \vspace*{3cm}
    \LARGE \@title \par
    \vspace{2cm}
    \textit{\small Autor:}\par
    \normalsize \@author\par \normalsize
    \vspace{3cm}
    \textit{\small Prowadzący:}\par
    Dr inż. Arkadiusz Grzybowski\par
    \vspace{2cm}
    Wydział Elektroniki\\ III rok\\ Pn TN 11.15 - 13.00\par
    \vspace{4cm}
    \small \@date
  \end{center}
\end{titlepage}
}
\makeatother


\begin{document}
\maketitle
  \section{Cel laboratorium}
  \paragraph{}
  Celem ćwiczenia jest zapoznanie się z oprogramowaniem GnuPG do szyfrowania i odpisywania dokumentów elektronicznych oraz zarządzaniem kluczami. W ramach ćwiczenia student zapoznaje sie z: podstawowymi algorytmami kryptograficznymi, tworzeniem, eksportem i importem kluczy, podpisywaniem e-maili oraz plików, budowaniem sieci zaufania.


  \section{Generowanie kluczy}
  \paragraph{}
  Klucz GPG można utworzyć z lini komend wydając polecenie \texttt{gpg --gen-key}.
  Podczas generowania program pyta użytkownika o szereg opcji:

  \begin{itemize}
    \item Typ szyfrowania - RSA, DSA
    \item Długość klucza - od 1024 do 4096 bitów
    \item Czas ważności klucza
    \item Dane właściciela klucza - imie i nazwisko, adres email oraz opcjonalny komentarz
  \end{itemize}

  \paragraph{}
  Ostatnim etapem generowania klucza jest dwukrotne podanie hasła zabezpieczającego klucz.

  \paragraph{}
  Rysunek 1 pokazuje generowanie klucza z szyfrowaniem RSA, o długości klucza 1024 bitów, ważnego przez 4 dni.

  \begin{figure}[h!]
    \begin{center}
      \includegraphics[width=\textwidth]{img/1.png}
      \caption{Generowanie klucza GPG}
    \end{center}
  \end{figure}

  \newpage
  \paragraph{}
  \newpage

  \section{Dodawanie, usuwanie i modyfikacja kluczy}
  \paragraph{}
  W celu wyświetlenia listy wygenerowanych kluczy można użyć polecenia \texttt{gpg --list-keys}.

  \begin{figure}[h!]
    \begin{center}
      \includegraphics[width=\textwidth]{img/2.png}
      \caption{Wyświetlenie listy kluczy}
    \end{center}
  \end{figure}

  \paragraph{}
  Edycji klucza można dokonać przy pomocy polecenia \texttt{gpg --edit-key NAZWA}. Rysunek 3 pokazuje proces edycji klucza o nazwie $"Tymon\ Tobolski\ <181037@student.pwr.wroc.pl>"$. W tym wypadku został zmieniony termin ważności klucza na 2 miesiące przy użyciu komendy \texttt{expire}. W celu sprawdzenia wszysktich dostępnych komend można wykorzystać polecenie \texttt{help} Przy zapisie zmian program wymaga podania hasła zabezpieczającego klucz.


  \begin{figure}[h!]
    \begin{center}
      \includegraphics[width=\textwidth]{img/3.png}
      \caption{Edycja klucza}
    \end{center}
  \end{figure}

  \newpage
  \paragraph{}
  \newpage

  \paragraph{}
  W celu usunięcia klucza należy najpierw usunąć klucz prywatny przy pomocy polecenia \texttt{gpg --delete-secret-key NAZWA}, a następnie usunąc klucz publiczny za pomocą komendy \texttt{gpg --delete-key NAZWA}. Samo usunięcie klucza prywatnego nie powoduje skosowania użytkownika. Nie jest możliwe usunięcie tylko klucza publicznego.


  \begin{figure}[h!]
    \begin{center}
      \includegraphics[width=\textwidth]{img/4.png}
      \caption{Usunięcie klucza}
    \end{center}
  \end{figure}

  \newpage

  \section{Unieważnienie klucza}
  \paragraph{}
  Unieważnienie klucza możliwe jest w trybie edycji za pomocą komendy \texttt{revkey}. Rysunek 5 przedstawia przykład unieważnienia klucza za pomocą programu gpg. Podczas procesu uniważnienia można opcjonalnie podać przyczyne oraz komentarz. Wymagane jest również podanie hasła zabezpieczającego dany klucz.


  \begin{figure}[h!]
    \begin{center}
      \includegraphics[width=\textwidth]{img/5.png}
      \caption{Unieważnienie klucza}
    \end{center}
  \end{figure}

  \section{Eksport i import kluczy}
  \paragraph{}
  Eksport klucza publicznego jak i prywatnego sprowadza się do wywołania 2 komend: \texttt{gpg --export -a NAZWA} dla klucza publicznego oraz \texttt{gpg --export-secret-key -a} dla klucza prywatnego. Rysunki 6 i 7 pokazują wynik działania tych komend.

  \begin{figure}[h!]
    \begin{center}
      \includegraphics[width=\textwidth]{img/6.png}
      \caption{Eksport klucza publicznego do pliku}
    \end{center}
  \end{figure}

  \begin{figure}[h!]
    \begin{center}
      \includegraphics[width=\textwidth]{img/7.png}
      \caption{Eksport klucza prywatnego do pliku}
    \end{center}
  \end{figure}

  \newpage


  \paragraph{}
  Po wyeksportowaniu klucza publicznego do pliku, można go przesłać innemu użytkownikowi. Może on wtedy zaimportować klucz i podpisać go za pomocą swojego własnego. W celu podpisania klucza przy użyciu programu gpg należy wejść w tryb edycji klucza, a następnie wywołać polecenie \texttt{sign}. Następnie należy podać hasło zabezpieczające klucz, którym podpisany będzie klucz .Rysunki 8 i 9 przedstawiają operacje importu oraz podpisywania klucza.

  \begin{figure}[h!]
    \begin{center}
      \includegraphics[width=\textwidth]{img/8.png}
      \caption{Import klucza z pliku}
    \end{center}
  \end{figure}

  \begin{figure}[h!]
    \begin{center}
      \includegraphics[width=\textwidth]{img/9.png}
      \caption{Podpisanie klucza}
    \end{center}
  \end{figure}

  \paragraph{}
  Budowanie sieci zaufania może okazać się czasochłonne, ze względu na konieczność wymiany kluczy między użytkownikami.

  \newpage

  \section{Wykorzystanie serwerów kluczy}
  \paragraph{}
  Wysłanie klucza na serwer wymaga podania adresu serwera, odbywa się za pomocą komendy \texttt{gpg --keyserver SERWER --send-keys ID\_KLUCZA}. Rysunek 10 przedstawia proces wysłania klucza na serwer pgp.mit.edu.

  \begin{figure}[h!]
    \begin{center}
      \includegraphics[width=\textwidth]{img/10.png}
      \caption{Przesłanie klucza na serwer pgp.mit.edu}
    \end{center}
  \end{figure}

  \paragraph{}
  Serwer kluczy udostępnia możliwość wyszukiwania kluczy na dwa sposoby: za pomocą identyfikatora klucza (\texttt{gpg --keyserver SERWER --search-key ID\_KLUCZA}) lub poprzed adres email (\texttt{gpg --keyserver SERWER --search-key EMAIL}). Rysunek 11 przedstawia obie metody wyszukiwania klucza na serwerze pgp.mit.edu.

  \begin{figure}[h!]
    \begin{center}
      \includegraphics[width=\textwidth]{img/11.png}
      \caption{Wyszukiwanie klucza na serwerze pgp.mit.edu}
    \end{center}
  \end{figure}

  \paragraph{}
  Serwery kluczy znacznie usprawniają wymiane kluczy między użytkownikami. Każdy może zapytać serwer o klucz publiczny podanego adresu email, nie ma potrzebu wysyłania go do każdego użytkownika z osobna.

  \section{Szyfrowanie, deszyfrowanie, podpisywanie i weryfikacja podpisu plików}
  \paragraph{}
  Podpisywanie plików dobywa się za pomocą komendy \texttt{gpg --clearsign PLIK}. Podpis pliku wymaga podania hasła zabezpieczającego klucz. Po podpisaniu pliku zostaje utworzony plik z podpisem o rozszerzeniu .asc  Przyład takiego podpisu prezentuje Rysunek 12.

  \begin{figure}[h!]
    \begin{center}
      \includegraphics[width=\textwidth]{img/12.png}
      \caption{Podpisywanie pliku}
    \end{center}
  \end{figure}

  \paragraph{}
  Weryfikacji podpisu można dokonać przy pomocy komendy \texttt{gpg --verify PLIK\_ASC}. Przykładowa operacja zaprezentowana jest na Rysunku 13.

  \begin{figure}[h!]
    \begin{center}
      \includegraphics[width=\textwidth]{img/13.png}
      \caption{Weryfikacja podpisu pliku}
    \end{center}
  \end{figure}

  \paragraph{}
  W celu zaszyfrowania wiadomości należy wywołać komende \texttt{gpg --recipient ODBIORCA --output PLIK\_GPG --encrypt PLIK}. Po zaszyfrowaniu powstaje plik .gpg. Przyład szyfrowania wiadomości znajduje się na Rysunku 14.

  \begin{figure}[h!]
    \begin{center}
      \includegraphics[width=\textwidth]{img/14.png}
      \caption{Szyfrowanie wiadomości}
    \end{center}
  \end{figure}

  \paragraph{}
  Deszyfracja wiadomości sprowadza się do wykonania komendy \texttt{gpg --decrypt-files PLIK\_GPG}. Ta operacja wymaga podania hasła zabezpieczającego klucz. Przykład użycia obrazuje Rysunek 15.

  \begin{figure}[h!]
    \begin{center}
      \includegraphics[width=\textwidth]{img/15.png}
      \caption{Szyfrowanie wiadomości}
    \end{center}
  \end{figure}

  \section{Szyfrowanie, deszyfrowanie, podpisywanie i weryfikacja podpisu e-mail}
  \paragraph{}
  W celu podpisania oraz szyfrowania wiadomości e-mail w programie Mozilla Thunderbird wystarczy wybrać odpowiednią opcje z menu OpenPGP podczas tworzenia nowej wiadomości. Umiejscowowienie wspomnianych opcji prezentuje Rysunek 16.

  \begin{figure}[h!]
    \begin{center}
      \includegraphics[width=\textwidth]{img/16.png}
      \caption{Podpisywanie oraz szyfrowanie nowej wiadomości}
    \end{center}
  \end{figure}

  \paragraph{}
  Program pocztowy Mozilla Thunderbird umożliwia także automatyczną weryfikację tożsamości nadawcy oraz deszyfrację wiadomości przychodzących.

  \section{Różnice między PGP/inline a PGP/MIME}
  \paragraph{}
  Domyślną metodą załączania zaszyfrowanej wiadomości jest PGP/inline, w której to informację o użytym kluczu sa przechowywane wewnątrz wiadomości. Wadą tej metody są problemy z szyfrowaniem niestandardowych znaków (spoza tablicy ASCII) oraz trudności z szyfrowaniem załączników. Ponadto programy pocztowe, które nie wspierają PGP wyświetlają wiadomości jako niezrozumiały tekst. Metoda PGP/MIME rozwiązuje powyższe problemy przenosząc informacje o kluczu do nagłowka wiadomości.

  \section{Wnioski}
  \paragraph{}
  Podpisywanie i szyfrowanie plików oraz wiadomości e-mail pozwala na zapobieganie przedostania się poufnych informacji do osób nieupoważnionych. Nowoczesne programy pocztowe wspierają obsługę OpenPGP, dlatego też generowanie kluczy oraz zabezpieczanie wiadomości nie jest procesem skomplikowanych dla użytkownika.



\end{document}
