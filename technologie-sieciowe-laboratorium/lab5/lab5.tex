\documentclass[wide,a4paper,titlepage,12pt] {article}
\usepackage{polski}
\usepackage{float}
\usepackage[utf8]{inputenc}
\usepackage{listings}
\usepackage{slashbox}
\usepackage[table]{xcolor}
\usepackage{graphicx,pdflscape}
\usepackage{placeins}
\usepackage{reportshelper}

\title{Technologie sieciowe 2}
\author{Tymon Tobolski (181037)\\ Jacek Wieczorek (181043)}

% Title page layout (fold)
\makeatletter
\renewcommand{\maketitle}{
\begin{titlepage}
  \begin{center}
    \vspace*{3cm}
    \LARGE \@title \par
    \vspace{2cm}
    \textit{\small Autor:}\par
    \normalsize \@author\par \normalsize
    \vspace{3cm}
    \textit{\small Prowadzący:}\par
    Dr inż. Arkadiusz Grzybowski\par
    \vspace{2cm}
    Wydział Elektroniki\\ III rok\\ Pn TN 11.15 - 13.00\par
    \vspace{4cm}
    \small \@date
  \end{center}
\end{titlepage}
}
\makeatother


\begin{document}
\maketitle
\section{Cel laboratorium}
\paragraph{}
Celem laboratorium było zapoznanie się z podstawowymi problemami związanymi z budową bezprzewodowej sieci LAN pracującej w standardzie IEEE 802.11 b/g/n.

\section{Konfiguracja sieci typu ad hoc} % (fold)
\label{sec:}
Konfiguracja sieci ad hoc w systemie Windows 7 jest intuicyjna i nie stanowi żadnego problemu. 

\putpicturescale{img/j3.PNG}{Skonfigurowana sieć typu ad hoc}{0.8}{fig:adhocconf}

\paragraph{} % (fold)
\label{par:}
Rysunek \ref{fig:adhocconf} przedstawia podstawową konfigurację sieci bezprzewodowej typu ad hoc, której nadaliśmy nazwę \textit{WieczorekTobolski} oraz takei samo SSID. 
\paragraph{} % (fold)
\label{par:}
W celu weryfikacji poprawności połączenia, wykonana została komenda ping:
% paragraph  (end)

\putpicturescale{img/j1.PNG}{Komenda ping do użytkownika Tymon Tobolski}{0.8}{fig:adhocping1}

\putpicturescale{img/t2-adhoc-ping.PNG}{Komenda ping do użytkownika Jacek Wieczorek}{0.8}{fig:adhocping2}

\paragraph{} % (fold)
\label{par:}
Rysunki \ref{fig:adhocping1} i \ref{fig:adhocping2} przedstawiają pomyślne wykonanie komendy ping, weryfikujące połączenie pomiędzy dwoma użytkownikami.
% paragraph  (end)
\newpage
\paragraph{} % (fold)
\label{par:}
W celu zabezpieczenia sieci ad hoc przed nieporządanym dostępem, zabezpieczona została hasłem:
\putpicturescale{img/j4.PNG}{Zabezpieczona sieć typu ad hoc}{0.6}{fig:adhocpass}

\putpicturescale{img/t3-adhoc-wrong-pass.PNG}{Źle wprowadzone hasło}{0.5}{fig:adhocwrongpass}

\paragraph{} % (fold)
\label{par:}
Na Rysunku \ref{fig:adhocpass} przedstawiona została konfiguracja zabezpieczenia sieci hasłem dostępu, natomiast Rysunek \ref{fig:adhocwrongpass} przedstawia próbę uzyskania nieautoryzowanego dostępu do sieci, która zakończyła się niepowodzeniem.

\end{document}