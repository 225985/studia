\documentclass[wide,a4paper,titlepage,12pt] {article}
\usepackage{polski}
\usepackage[utf8]{inputenc}
\usepackage{listings}
\usepackage{slashbox}
\usepackage[table]{xcolor}
\usepackage{graphicx,pdflscape}
\usepackage{placeins}
\usepackage{reportshelper}


\title{Urządzenia peryferyjne}
\author{Tymon Tobolski (181037)\\ Jacek Wieczorek (181043)}

% Title page layout (fold)
\makeatletter
\renewcommand{\maketitle}{
\begin{titlepage}
  \begin{center}
    \vspace*{3cm}
    \LARGE \@title \par
    \vspace{2cm}
    \textit{\small Autor:}\par
    \normalsize \@author\par \normalsize
    \vspace{3cm}
    \textit{\small Prowadzący:}\par
    Dr inż. Jacek Mazurkiewicz \par
    \vspace{2cm}
    Wydział Elektroniki\\ III rok\\ Pn 8.15 - 11.00\par
    \vspace{4cm}
    \small \@date
  \end{center}
\end{titlepage}
}
\makeatother



\begin{document}
\maketitle

\section{Cel laboratorium}
\paragraph{}
Celem laboratorium było zapoznanie się z zasdą pracy i obsługą skanera płaskiego. W tym celu wykorzystana została biblioteka WIA 1.0 i język programowania $C\#$

\section{Parametry skanowania}
\paragraph{} % (fold)
Poza samym zeskanowaniem obrazu, udało nam sie również zmeiniać podstawowe parametry skanowania, takie jak:
\begin{itemize}
    \item czarnobiałe lub kolorowe
    \item rozdzielczość (DPI)
    \item kontrast
    \item jasność
\end{itemize}

\subsection{Rozdzielczość}
\paragraph{}
DPI (dots per inch) - liczba plamek przypadająca na cal, używane jako miara rozdzielczości drukarek, ploterów, skanerów itp. 

\subsection{Kontrast}
\paragraph{} % (fold)
\label{par:}
TODO

\subsection{Jasność}
\paragraph{} % (fold)
\label{par:}
TODO
% paragraph  (end)

\section{Implementacja}
\paragraph{} % (fold)
\label{par:}
Program do obsługi skanera płaskiego napisany został w języku $C\#$ z wykorzystaniem biblioteki WIA 1.0 .

\paragraph{} % (fold)
\label{}
W celu wyboru urządzenia skanującego skorzystaliśmy ze standardowego okna dialogowego dostepnego w bibliotece.

\putcode{code/p1.cs}{c++} 

\paragraph{}
Do inicializacji urządzenia i określenia parametrów skanowania : 
\putcode{code/p2.cs}{c++} 

\paragraph{} % (fold)
\label{par:}
Proces skanowania :
\putcode{code/p3.cs}{c++} 

\paragraph{} % (fold)
\label{par:}
Zapis obrazu do pliku :
\putcode{code/p4.cs}{c++}

\paragraph{} % (fold)
\label{par:}
Przykładowe okno programu :
\putpicture{img/program.PNG}{Przykładowe okno porgramu}{\textwidth}{}
% paragraph  (end)

% paragraph  (end)

\section{Wnioski}
Napisanie programu do obsługi skanera płaskiego w wykorzystaniem biblioteki WIA nie jest zadaniem trudnym. Jedynym problemem z jakim spotkaliśmy się podczas implementacji było znalezienie odpowiednich wartości stałych by ustawić parametry skanowania.
\paragraph{}
\end{document}